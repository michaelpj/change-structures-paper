% Toggle comments for preamble and topmatter to typeset in ACM style

%\input{preamble-standard}
%% For double-blind review submission, w/o CCS and ACM Reference (max submission space)
\documentclass[sigplan,9pt,review,anonymous,natbib=false]{acmart}\settopmatter{printfolios=true,printccs=false,printacmref=false}
%% For double-blind review submission, w/ CCS and ACM Reference
%\documentclass[sigplan,9pt,review,anonymous,natbib=false]{acmart}\settopmatter{printfolios=true}
%% For single-blind review submission, w/o CCS and ACM Reference (max submission space)
%\documentclass[sigplan,9pt,review,natbib=false]{acmart}\settopmatter{printfolios=true,printccs=false,printacmref=false}
%% For single-blind review submission, w/ CCS and ACM Reference
%\documentclass[sigplan,9pt,review,natbib=false]{acmart}\settopmatter{printfolios=true}
%% For final camera-ready submission, w/ required CCS and ACM Reference
%\documentclass[sigplan,9pt,natbib=false]{acmart}\settopmatter{}

%% Conference information
%% Supplied to authors by publisher for camera-ready submission;
%% use defaults for review submission.
\acmConference[PL'17]{ACM SIGPLAN Conference on Programming Languages}{January 01--03, 2017}{New York, NY, USA}
\acmYear{2017}
\acmISBN{} % \acmISBN{978-x-xxxx-xxxx-x/YY/MM}
\acmDOI{} % \acmDOI{10.1145/nnnnnnn.nnnnnnn}
\startPage{1}

%% Copyright information
%% Supplied to authors (based on authors' rights management selection;
%% see authors.acm.org) by publisher for camera-ready submission;
%% use 'none' for review submission.
\setcopyright{none}
%\setcopyright{acmcopyright}
%\setcopyright{acmlicensed}
%\setcopyright{rightsretained}
%\copyrightyear{2017}           %% If different from \acmYear

%% Bibliography style
\bibliographystyle{ACM-Reference-Format}
%% Citation style
%\citestyle{acmauthoryear}  %% For author/year citations
%\citestyle{acmnumeric}     %% For numeric citations
%\setcitestyle{nosort}      %% With 'acmnumeric', to disable automatic
                            %% sorting of references within a single citation;
                            %% e.g., \cite{Smith99,Carpenter05,Baker12}
                            %% rendered as [14,5,2] rather than [2,5,14].
%\setcitesyle{nocompress}   %% With 'acmnumeric', to disable automatic
                            %% compression of sequential references within a
                            %% single citation;
                            %% e.g., \cite{Baker12,Baker14,Baker16}
                            %% rendered as [2,3,4] rather than [2-4].


%%%%%%%%%%%%%%%%%%%%%%%%%%%%%%%%%%%%%%%%%%%%%%%%%%%%%%%%%%%%%%%%%%%%%%
%% Note: Authors migrating a paper from traditional SIGPLAN
%% proceedings format to PACMPL format must update the
%% '\documentclass' and topmatter commands above; see
%% 'acmart-pacmpl-template.tex'.
%%%%%%%%%%%%%%%%%%%%%%%%%%%%%%%%%%%%%%%%%%%%%%%%%%%%%%%%%%%%%%%%%%%%%%


%% Some recommended packages.
\usepackage{booktabs}   %% For formal tables:
                        %% http://ctan.org/pkg/booktabs
\usepackage{subcaption} %% For complex figures with subfigures/subcaptions
                        %% http://ctan.org/pkg/subcaption


\usepackage[utf8]{inputenc}
\usepackage{amsmath}
\usepackage{amssymb}
\usepackage{amsthm}
\usepackage{mathtools}
\usepackage{stmaryrd}
\usepackage{thmtools}
\usepackage{todonotes}
\usepackage{etoolbox}
\usepackage{appendix}

% Packages where the order matters
\usepackage[safeinputenc,natbib=true]{biblatex}
\usepackage{cleveref}
\usepackage{hyperref}

\newcommand{\todoall}[1]{\todo[inline,color=black!30,author=All]{#1}}
\newcommand{\todokcg}[1]{\todo[inline,color=pink!60,author=Katriel]{#1}}
\newcommand{\todompj}[1]{\todo[inline,color=yellow!40,author=Michael]{#1}}
\newcommand{\todomario}[1]{\todo[inline,color=blue!40,author=Mario]{#1}}

% Theorem styles

\theoremstyle{plain}
\newtheorem{thm}{Theorem}

\theoremstyle{definition}
\newtheorem{prop}[thm]{Proposition}

\theoremstyle{remark}
\newtheorem{claim}[thm]{Claim}

\theoremstyle{remark}
\newtheorem{corollary}[thm]{Corollary}

\theoremstyle{remark}
\newtheorem{rem}[thm]{Remark}

\theoremstyle{definition}
\newtheorem{defn}[thm]{Definition}

% Notation

% General
\newcommand{\defeq}{\coloneqq}
\newcommand{\cat}[1]{\mathbf{#1}}
\newcommand{\equalizer}[2]{Eq(#1, #2)}
\newcommand{\powerset}[1]{\mathcal{P}(#1)}
\newcommand{\denote}[1]{\llbracket #1 \rrbracket}
\newcommand{\ev}{\operatorname{ev}}
\newcommand{\id}{\operatorname{id}}
\newcommand{\doubleplus}{\ensuremath{\mathbin{+\mkern-10mu+}}}
\newcommand{\pair}[2]{\left\langle {#1}, {#2} \right\rangle}
\newcommand{\into}[0]{\rightarrow}

% Change structures
% Generic "thing in a circle" operator
\makeatletter
\newcommand\cplussym
{
  \mathpalette\@incircbin
}
\newcommand\@incircbin[2]
{
  \mathbin
  {
    \ooalign{\hidewidth$#1#2$\hidewidth\crcr$#1\bigcirc$}%
  }
}
\makeatother

\newcommand{\cplus}{\oplus}
\newcommand{\cpluss}{\boxplus}
\newcommand{\cplusss}{\odot}
\newcommand{\cplusvee}{\cplussym{\vee}}
\newcommand{\cminus}{\ominus}

\newcommand{\splus}{\cdot}
\newcommand{\mzero}{\mathbf{0}}

\newcommand{\cstruct}[3]{(#1,#2,#3)}
\newcommand{\cstr}[1]{\hat{#1}}
\newcommand{\changes}[1]{\Delta #1}
\newcommand{\change}[1]{\delta #1}

\newcommand{\discrete}{\emptyset}

\newcommand{\derive}[1]{#1'}
\newcommand{\supderive}[1]{#1_\uparrow}
\newcommand{\supderiveM}[1]{#1_{\uparrow\uparrow}}
\newcommand{\subderive}[1]{#1_\downarrow}
\newcommand{\subderiveM}[1]{#1_{\downarrow\downarrow}}

\newcommand{\monotoneDerive}[1]{#1'^{M}}

\newcommand{\NN}{\mathbb{N}}
\newcommand{\ZZ}{\mathbb{Z}}

\newcommand{\kernel}{\sim}
\newcommand{\kernelOrder}{\leq_\kernel}

% Some kinds of function spaces
\newcommand{\ra}{\rightarrow}
\newcommand{\Ra}{\Rightarrow}
\newcommand{\ptfunc}{\Rightarrow_{pt}}

% Algebra
\newcommand{\reach}{\operatorname{Reach}}
\newcommand{\direct}{\operatorname{Direct}}
\newcommand{\superpose}{{\bowtie}}
\newcommand{\curry}[1]{\Lambda{#1}}
\newcommand{\exponential}[2]{{#1} \Rightarrow {#2}}

% Orders
\newcommand{\reachOrder}{\leq_R}
\newcommand{\changeOrder}{\leq_\Delta}
\newcommand{\fineOrder}{\leq_D}
\newcommand{\minusOrder}{\leq_\cminus}

% Lattices
\newcommand{\twist}{\cplus_{\bowtie}}
\newcommand{\disjointTimes}{\bowtie}
\newcommand{\updiff}{\Delta}
\newcommand{\downdiff}{\nabla}

% Fixpoints
\newcommand{\fixpoint}{\mathbf{fix}}
\newcommand{\lfp}{\mathbf{lfp}}
\newcommand{\adjust}{\mathbf{adjust}}
\newcommand{\iter}{\mathbf{iter}}
\newcommand{\nextiter}{\mathbf{recur}}

% Datalog
\newcommand{\Formula}{\mathrm{Formula}}
\newcommand{\Rel}{\cat{Rel}}
\newcommand{\universalRel}{\mathcal{U}}
\newcommand{\consq}{\mathcal{I}}


\newif\ifproofs
% Comment out to disable proofs
%\proofstrue

\addbibresource{paper.bib}

\begin{document}

%\author{
  Mario Alvarez Picallo\\
  University of Oxford
  \and
  Alex Eyers-Taylor\\
  Semmle Ltd.
  \and
  Michael Peyton Jones\\
  Semmle Ltd.
  \and
  Luke Ong\\
  University of Oxford
}

%% Author information
%% Contents and number of authors suppressed with 'anonymous'.
%% Each author should be introduced by \author, followed by
%% \authornote (optional), \orcid (optional), \affiliation, and
%% \email.
%% An author may have multiple affiliations and/or emails; repeat the
%% appropriate command.
%% Many elements are not rendered, but should be provided for metadata
%% extraction tools.

\author{Alex Eyers-Taylor}
\affiliation{
  \position{Research Engineer}
  \institution{Semmle Ltd}            %% \institution is required
  \country{United Kingdom}                    %% \country is recommended
}
\email{alex@semmle.com}          %% \email is recommended

\author{Michael Peyton Jones}
\affiliation{
  \position{Research Engineer}
  \institution{Semmle Ltd}            %% \institution is required
  \country{United Kingdom}                    %% \country is recommended
}
\email{michael@semmle.com}          %% \email is recommended

\author{Mario Alvarez Picallo}
\affiliation{
  \position{PhD Student}
  \department{Computer Science}
  \institution{University of Oxford}            %% \institution is required
  \country{United Kingdom}                    %% \country is recommended
}
\email{mario.alvarez-picallo@cs.ox.ac.uk}          %% \email is recommended


%% 2012 ACM Computing Classification System (CSS) concepts
%% Generate at 'http://dl.acm.org/ccs/ccs.cfm'.
\begin{CCSXML}
<ccs2012>
<concept>
<concept_id>10011007.10011006.10011008</concept_id>
<concept_desc>Software and its engineering~General programming languages</concept_desc>
<concept_significance>500</concept_significance>
</concept>
<concept>
<concept_id>10003456.10003457.10003521.10003525</concept_id>
<concept_desc>Social and professional topics~History of programming languages</concept_desc>
<concept_significance>300</concept_significance>
</concept>
</ccs2012>
\end{CCSXML}

\ccsdesc[500]{Software and its engineering~General programming languages}
\ccsdesc[300]{Social and professional topics~History of programming languages}
%% End of generated code

%% Keywords
%% comma separated list
\keywords{incremental computation, dcpos, fixpoints, Datalog}  %% \keywords are mandatory in final camera-ready submission


\begin{abstract}
  Incremental computation has recently been studied using the concepts of \emph{change
  structures} and \emph{derivatives} of programs. The derivative of a program allows updating the output
  of a program based on a change to its input.

  We generalise change structures to a Cartesian closed category of \emph{change actions},
  and study their algebraic properties. We develop change actions for several common structures
  in computer science, including directed-complete partial orders and Boolean algebras.

  We then show how to compute derivatives of fixpoints, leading to a
  generic account of incremental \emph{evaluation} of Datalog,
  as well as incremental \emph{updating} of evaluated Datalog programs.
\end{abstract}

\title{Fixing incremental computation}
\subtitle{Derivatives for dcpos, fixpoints, and the semantics of Datalog}

\maketitle

\section{Introduction}
\label{sec:intro}

Consider the following classic Datalog program, which computes the transitive
closure of an edge relation $e$:
\begin{align*}
  tc(x, y) &\leftarrow e(x, y)\\
  tc(x, y) &\leftarrow e(x, z) \wedge tc(z, y)
\end{align*}

The semantics of Datalog tells us that the denotation of this program is the
least fixpoint of the rule $tc$. Kleene's Theorem tells us that we can reach
this by repeatedly applying the rule, starting from the empty relation. For example, supposing
that $e = \{ (1, 2), (2, 3), (3, 4) \}$:
\begin{center}
  \begin{tabular} {p{3.5em} p{10em} p{10em}}
    Iteration & Newly deduced facts & Accumulated data in $tc$ \\
    \toprule
    1 & $\{ (1, 2), (2, 3), (3, 4) \}$ & $\{ (1, 2), (2, 3), (3, 4) \}$\\
    2 & $\{ (1, 2), (2, 3), (3, 4),$ $(1, 3), (2, 4) \}$ & $\{ (1, 2), (2, 3), (3, 4),$ $(1, 3), (2, 4) \}$\\
    3 & $\{ (1, 2), (2, 3), (3, 4),$ $(1, 3), (2, 4), (1, 4),(1, 4) \}$ & $\{ (1, 2), (2, 3), (3, 4),$ $(1, 3), (2, 4), (1, 4) \}$\\
    4 & (as above) & (as above) \\
    \bottomrule
  \end{tabular}
\end{center}
\medskip

And then we have reached fixpoint, so we are done.

However, this process is quite wasteful. We deduced $(1,2)$ at every iteration,
even though we had already deduced it in the first iteration. In fact, for a
chain of $n$ such edges we will deduce $O(n^2)$ facts along the way.

The standard improvement to this is ``semi-naive'' evaluation, where we transform
the program into a \emph{delta} program that only deduces the new facts at each
iteration, which we gradually accumulate \autocite[See][section
13.1]{abiteboul1995foundations}.
\begin{align*}
  \Delta tc_{0}(x, y) &\leftarrow e(x, y)\\
  \Delta tc_{i+1}(x, y) &\leftarrow e(x, z) \wedge \Delta_i tc(z, y)\\
  tc_{0}(x, y) &\leftarrow e(x, z)\\
  tc_{i+1}(x, y) &\leftarrow tc_{i}(x,y) \vee \Delta_{i+1} tc(x,y)
\end{align*}

\begin{center}
  \begin{tabular} {p{3.5em} p{8em} p{10em}}
    Iteration & $\Delta tc_i$ & $tc_i$ \\
    \toprule
    1 & $\{ (1, 2), (2, 3), (3, 4) \}$ & $\{ (1, 2), (2, 3), (3, 4) \}$\\
    2 & $\{ (1, 3), (2, 4) \}$ & $\{ (1, 2), (2, 3), (3, 4),$ $(1, 3), (2, 4) \}$\\
    3 & $\{ (1, 4) \}$ & $\{ (1, 2), (2, 3), (3, 4),$ $(1, 3), (2, 4), (1, 4) \}$\\
    4 & $\{ \}$ & (as above) \\
    \bottomrule
  \end{tabular}
\end{center}
\medskip

This is much better \textemdash we have turned a quadratic computation into a linear one.

But the delta rule translation works only for traditional Datalog. It is common to
liberalise the formula syntax with additional features, such as disjunction,
existential quantification, negation, and aggregates.\footnote{ See, for
  example, \autocites(LogiQL)(){logicbloxWebsite}{halpin2014logiql},
  \autocites(Datomic)(){datomicWebsite},
  \autocites(Souffle)(){souffleWebsite}{scholz2016fast}, and
  \autocites(DES)(){saenz2011deductive}, which between them have all of these
  features and more. } 
Then we can write programs like the following, where we compute whether all the
nodes in a subtree given by $child$ have some property $p$:
\begin{align*}
  treeP(x) &\leftarrow p(x) \wedge \neg \exists y . (child(x,y) \wedge \neg treeP(y))
\end{align*}

Here the combination of negation and explicit existential quantification amounts
to recursion through a \emph{universal} quantifier. We would
like to be able to use semi-naive evaluation for this rule too, but the simple delta
transformation does not produce a correct incremental program, and it is unclear how to extend it (and the
correctness proof) to handle such cases.

This is of more than theoretical interest \textemdash the research
in this paper was carried out at Semmle, which
makes heavy use of a commercial Datalog implementation
\autocites{semmleWebsite}{avgustinov2016ql}{sereni2008adding}{schafer2010type}.
Semmle's implementation includes parity-stratified negation\footnote{Parity-stratified negation means that recursive calls must
  appear under an even number of negations. This ensures that the rule remains
  monotone, so the least fixpoint still exists and can be found via Kleene's
  theorem.}, recursive aggregates \autocite{demoor2013aggregates}, and other non-standard
features, so we are faced with a dilemma: either abandon the new language
features, or abandon incremental computation.

There is a piece of folkloric knowledge in the Datalog community that hints at a
solution: the semi-naive translation of a rule corresponds to the
\emph{derivative} of that rule \autocites{bancilhon1986naive}[section
3.2.2]{bancilhon1986amateur}. The idea of performing incremental computation using derivatives has been
studied recently by \textcite{cai2014changes}, who give an account using
\emph{change structures}. They use this to provide a framework for incrementally evaluating lambda calculus programs.

However, \citeauthor{cai2014changes}'s work isn't directly applicable to Datalog: the tricky part
of Datalog's semantics are recursive definitions and the need for the \emph{fixpoints}, and we need some additional theory to tell us how to
handle incremental evaluation of fixpoint computations.

This paper bridges that gap.

\subsection{Contributions}

We start by generalizing change structures to
\emph{change actions} (\cref{sec:changeActions}). Change actions are weaker than change structures, but
have nice categorical properties (\cref{sec:category}), and exist for more structures (\cref{sec:moreStructures}).

We then show how to compute fixpoints incrementally, and also how to perform
incremental updates of already computed fixpoint expressions, given a change to
the function of which the fixpoint is taken (\cref{sec:fixpoints}).

Finally, we put all this together into a generic account of incremental
computation and update of Datalog programs, showing how to handle 
language constructs such as negation, disjunction, and additional well-behaved
extensions (\cref{sec:datalog}). This provides the world's first incremental
evaluation and update mechanism for Datalog that can handle these features. Moreover, the structure of the
proof is modular, and can accommodate arbitrary additional
formula constructs (\cref{sec:extensions}).

We have omitted the proofs from this paper. Most of the results have routine
proofs, but the proofs of the more substantial results
(especially those in \cref{sec:fixpoints}) are included in an appendix.

\section{Change actions}
\label{sec:changeActions}

The core concept we will work with is a \emph{change action}. A change action is
a set along with a set of \emph{changes} that can be ``applied'' to elements of
the base set using an operator $\cplus$ (pronounced ``smush''). We also require
some structure on the changes themselves: they must form a \emph{monoid action}
on the base set.\footnote{The requirement that the change set be a monoid is convenient but in
  fact inessential: given any set with an action on the base set, we can take the
  free monoid over the action set to obtain a monoid action.}

\begin{defn}[Change actions]
  A \emph{change action} is defined as:

  \begin{displaymath}
    \cstr{A} \defeq \cstruct{A}{\changes{A}}{\cplus}
  \end{displaymath}

  where $A$ is a set, $\changes{A}$ is a monoid, and $\cplus : A \rightarrow
  \changes{A} \rightarrow A$ is a monoid action on $A$.

  We will call $A$ the base set, and $\changes{A}$ the change set of the change
  action. We will use $\splus$ for the monoid action of $\changes{A}$, and
  $\mzero$ for its identity element. We may abbreviate $\cstruct{A}{\changes{A}}{\cplus}$ to $\cstr{A}_\cplus$.
\end{defn}

The fact that the change set is a monoid action gives us a reason to think that
change actions are an adequate representation of changes: any subset of $A
\rightarrow A$ which is closed under composition can be
represented as a monoid action, so we are able to capture all of these as change actions.

The primary motivation for change actions is that they let us define
\emph{derivatives} for functions.

\begin{defn}[Derivatives]
  \label{def:derivative}
  A \emph{derivative} of a function $f: \cstr{A}_\cplus \rightarrow \cstr{B}_\cpluss$ is a function $\derive{f}: A \times \changes{A} \rightarrow
  \changes{B}$ such that
  \begin{displaymath}
    f(a \cplus \change{a}) = f(a) \cpluss \derive{f}(a, \change{a})
  \end{displaymath}

  A function which has a derivative is called
  \emph{differentiable}.\footnote{Note that we do not require that $\derive{f}(a,
    \change{a} \splus \change{b}) = \derive{f}(a, \change{a}) \splus \derive{f}(a
    \cplus \change{a}, \change{b})$. This is a natural condition, and almost all
    derivatives satisfy it, but we have not found it to be necessary so far.}
\end{defn}

Derivatives need not be unique in general, so we will speak of ``a''
derivative.\footnote{In several places we will need to pick an arbitrary
  derivative for some construction. In general this needs the Axiom of Choice,
  but in most practical cases we will want to have a computable derivative
  operator for our domain, which alleviates the problem.} A ``thin'' change
action \textemdash where $a \cplus \change{a} = a \cplus \change{b}$ implies $\change{a} =
\change{b}$ \textemdash has unique derivatives, but many change actions are not thin.
For example, because $\{0\} \cap \{1\} = \{0\}
\cap \{2\}$, $\cstruct{\mathcal{P}(\NN)}{\mathcal{P}(\NN)}{\cap}$ is not thin.

Derivatives capture the essence of incremental computation: given the value of a
function at a point, and a change to that point, they tell you how to compute
the new value of the function.

The derivative of a function can be computed compositionally, because derivatives satisfy the standard chain rule.

\begin{thm}[The Chain Rule]
  Let $f: \cstr{A}_\cplus \rightarrow \cstr{B}_\cpluss$, $g: \cstr{B}_\cpluss \rightarrow \cstr{C}_\cplusss$ be differentiable functions. Then $g \circ f$ is also
  differentiable, with derivative given by
  \begin{displaymath}
    \derive{(g \circ f)}(x, \change{x}) = \derive{g}\left(f(x), \derive{f}(x, \change{x})\right)
  \end{displaymath}
  or, in curried form
  \begin{displaymath}
    \derive{(g \circ f)}(x) = \derive{g}(f(x)) \circ \derive{f}(x)
  \end{displaymath}
\end{thm}

For our Datalog example, we are going to want a derivative of the semantics of
Datalog. As we will discuss later (\cref{sec:datalog}), we can see Datalog's
semantics as computing an n-tuple of relations. So we need a change action on
$\Rel^n$, and fortunately we can construct good change actions for Boolean
algebras in general (\cref{sec:booleanAlgebras}).

Here are some other recurring examples of change actions:
\begin{itemize}
  \item $\cstr{A}_\discrete \defeq \cstruct{A}{\emptyset}{\lambda(a, da). a}$,
    which we call the \emph{discrete} change action on any base set.
  \item $\cstr{A}_\Rightarrow \defeq \cstruct{A}{\exponential{A}{A}}{\ev}$, where $A$ is some
    category, $\exponential{A}{A}$ is the exponential object (which is a monoid
    under composition), and $\ev$ is the evaluation map
    (for example, functions and function application).
  \item $[A]$, the type of lists (or streams) of elements of type $A$, has a
    change action over concatenation ($\doubleplus$): $\cstruct{[A]}{[A]}{\doubleplus}$.
  \item $\{A\}$, the type of sets, has change actions $\cstruct{\{A\}}{\{A\}}{\cup}$,\\ $\cstruct{\{A\}}{\{A\}}{\cap}$.
  \item $\cstr{\NN} \defeq \cstruct{\NN}{\NN}{+}$, the natural numbers with itself as change
    set under the additive monoid.
\end{itemize}

Indeed, any monoid $(A, \splus)$ can be seen as a change action
$\cstruct{A}{A}{\splus}$, and the final three examples are instances of this. Many practical change actions
can be constructed in this way. In particular, for any change action $\cstruct{A}{\changes{A}}{\cplus}$,
$\cstruct{\changes{A}}{\changes{A}}{\splus}$ is also a change action. This means
that we don't have to do any extra work to talk about higher derivatives of
functions \textemdash we can always take $\changes{\changes{A}} = \changes{A}$.

Many other notions in computer science can be naturally understood in terms of change actions, \emph{e.g.} databases
and database updates, files and diffs, Git repositories and commits, even video compression
algorithms that encode a frame as a series of changes to the previous frame.

\subsection{Complete change actions and minus operators}

Complete change actions are an important class of change actions, because they
have changes between \emph{any} two values in the base set.

\begin{defn}[Complete change actions]
  A change action is \emph{complete} if for any $a, b \in A$, there is
  a change $\change{a} \in \changes{A}$ such that $a \cplus \change{a} = b$.
\end{defn}

Complete change actions have convenient ``minus operators'' that allow us to
compute the difference between two values.

\begin{defn}[Minus operator]
  A \emph{minus operator} is a function $\cminus: A \times A \rightarrow \changes{A}$ such that $a \cplus (b \cminus a) = b$.
\end{defn}

\begin{prop}[Completeness equivalences]
  Let $\cstr{A}$ be a change action. Then the following are equivalent:
  \begin{itemize}
    \item $\cstr{A}$ is complete.
    \item There is a minus operator on $\cstr{A}$.
    \item Any function $f: \cstr{B} \rightarrow \cstr{A}$ is differentiable.
  \end{itemize}
\end{prop}

This last property is of the utmost importance, since we are often concerned with the differentiability
of functions.

\begin{prop}[Minus derivative]
  Given a minus operator $\cminus$, and a function $f$, let
  \begin{displaymath}
    \derive{f}_\cminus(a, \change{a}) \defeq f(a \cplus \change{a}) \cminus f(a)
  \end{displaymath}

  Then $\derive{f}_\cminus$ is a derivative for $f$.
\end{prop}

\section{The category of change actions}
\label{sec:category}

We are arguing that change actions provide a good model for incremental
computation in general. That means we want to be able to easily construct change
actions over the wide variety of datatypes that we actually use in programming.

We can do this by showing that the category of change actions has the usual
useful constructions: products ($A \times B$), coproducts ($A + B$), and
exponentials ($A \rightarrow B$). These are the
building blocks from which we create our datatypes, and they will also be useful
for our proofs in \cref{sec:fixpoints}: in particular, the \emph{evaluation map}
$\ev$ that we get from exponentials will turn out to be differentiable, and its
derivative has a deep connection to the incremental evaluation of functions (\cref{prop:incrementalization}).

\begin{defn}[Category of change actions]
  We define the category $\cat{CAct}$ of change actions. The objects are
  change actions and the morphisms are differentiable functions. 
\end{defn}

In \cref{sec:moreStructures} we will look at subsets of $\cat{CAct}$ with
interesting properties. However, for simplicity we will keep working in $\cat{CAct}$
throughout, rather than defining subcategories.

\subsection{Equivalence with PreOrd}

There is a natural preorder on the base set of a change action, given by reachability
under the action:

\begin{defn}[Reachability preorder]
  $a \reachOrder b$ iff there is a $\change{a} \in \changes{A}$ such that $a \cplus
  \change{a} = b$.
\end{defn}

We can characterize many of the properties of change actions in terms of the reachability preorder,
which suggests a connection between $\cat{CAct}$ and the category of preorders, $\cat{PreOrd}$.

\begin{prop}
  A function is differentiable iff it is monotone with respect to the
  reachability preorder. 
\end{prop}

\begin{corollary}
  Two change actions are isomorphic iff their posets under the reachability
  preorder are isomorphic.
\end{corollary}

\begin{corollary}
  Any function from a discrete change action or into a complete change
  action is differentiable.
\end{corollary}

Indeed, the correspondence between a change action and its reachability preorder gives rise to
a (full and faithful) functor $\reach : \cat{CAct} \rightarrow \cat{PreOrd}$ that acts as the
identity on morphisms.

Conversely, any preorder $\leq$ on some set $A$ induces a change action
$\cstr{A}_\leq \defeq \cstruct{A}{\leq^\star}{\cplus_\star}$.
The action $\cplus_\star$ is defined as the extension of $\cplus_\leq$ to the free
monoid $\leq^\star$, where $\cplus_\leq$ is defined as:
\[
\begin{aligned}
   \cplus_\leq &: (A \times \leq) \rightarrow A&\\
   a \cplus_\leq (b, c) &\defeq
     \begin{cases}
     c&\text{ if $a = b$}\\
     a&\text{ otherwise}
     \end{cases}
\end{aligned}
\]

The mapping to $\cstr{A}_\leq$ gives rise to a (full and faithful) functor
$\direct : \cat{PreOrd} \rightarrow \cat{CAct}$, again acting as the identity on morphisms.

These two functors are in fact enough to give us an equivalence between the categories
$\cat{CAct}$ and $\cat{PreOrd}$.

\begin{thm}[name=Equivalence of $\cat{CAct}$ and $\cat{PreOrd}$, restate=preordEquivalence]
  \label{thm:preordEquivalence}
  The functor $\reach$ from $\cat{CAct}$ to $\cat{PreOrd}$ together with the
  functor $\direct$ in the opposite direction form an equivalence of categories.
\end{thm}
\ifproofs
\begin{proof}
  See \cref{prf:preordEquivalence}.
\end{proof}
\fi

Since $\cat{PreOrd}$ is a Cartesian closed category and has all limits and
colimits, this gives us a proof of the existence of limits, colimits, and exponentials in $\cat{CAct}$.

\begin{corollary}
  The category $\cat{CAct}$ has all limits, colimits and exponential objects.
\end{corollary}

In the case of complete change actions, this equivalence degenerates into an
equivalence with $\cat{Set}$, since all functions are differentiable.

\subsection{Explicit constructions}

Having shown that $\cat{CAct}$ is equivalent to $\cat{PreOrd}$ it may seem
redundant to give explicit recipes for constructing products, coproducts, and exponentials.
However, we can give constructions that are much nicer than the ones
which we get via $\cat{PreOrd}$, which is important for using them in
practical computation.

The products and coproducts are fairly straightforward.

\begin{prop}[name=Products, restate=products]
  \label{prop:products}
  Let $\cstr{A} = \cstruct{A}{\changes{A}}{\cplus}$ and $\cstr{B} =
  \cstruct{B}{\changes{B}}{\cpluss}$ be change actions.

  Then $\cstr{A} \times \cstr{B} \defeq \cstruct{A \times B}{\changes{A} \times
  \changes{B}}{\cplus \times \cpluss}$ is their categorical product.
\end{prop}
\ifproofs
\begin{proof}
  See \cref{prf:products}.
\end{proof}
\fi

\begin{prop}[name=Coproducts, restate=coproducts]
  \label{prop:coproducts}
  Let $\cstr{A} = \cstruct{A}{\changes{A}}{\cplus}$ and $\cstr{B} =
  \cstruct{B}{\changes{B}}{\cpluss}$ be change actions.

  Then $\cstr{A} + \cstr{B} \defeq \cstruct{A + B}{\changes{A} \times
  \changes{B}}{\cplusvee}$ is their categorical coproduct, with $\cplusvee$ defined as:
  \begin{align*}
    i_1 a \cplusvee (\change{a}, \change{b}) &\defeq \iota_1 (a \cplus \change{a})\\
    i_2 b \cplusvee (\change{a}, \change{b}) &\defeq \iota_2 (b \cplus \change{b})
  \end{align*}
\end{prop}
\ifproofs
\begin{proof}
  See \cref{prf:coproducts}.
\end{proof}
\fi

The exponential objects in the category $\cat{CAct}$ are difficult to work with,
because we need to ensure that $f \cplus \change{f}$ continues to be a
differentiable function. Under some
conditions, however, we can find a simpler representation.

\begin{defn}[Pointwise change actions]
  \label{def:pointwiseChanges}
  A change action $\cstr{B}$ is pointwise if every exponential object
  $\exponential{\cstr{A}}{\cstr{B}}$ is isomorphic to the pointwise change action
  $\cstruct{\cstr{A} \rightarrow \cstr{B}}{A \rightarrow \changes{B}}{\cplus_\rightarrow}$,
  where $(f \cplus_\rightarrow \change{f})(x) \defeq f(x) \cplus \change{f}(x)$ and the
  monoidal structure on $A \rightarrow \changes{B}$ is given by lifting
  the monoidal structure on $\changes{B}$ pointwise.
\end{defn}

This corresponds to the intuition about how the correspondence with
$\cat{PreOrd}$ should work: the preorder on monotone functions in $\cat{PreOrd}$ is
pointwise.

This pointwise change action is not well defined for all change actions, since
we require $f \cplus_\rightarrow \change{f}$ to be \emph{differentiable}, which
may not be true for all pointwise changes $\change{f}$.
Fortunately, there are useful subcategories of $\cat{CAct}$ formed of pointwise change actions.

\begin{prop}[name=Pointwise change actions, restate=pointwiseChangeActions]
  \label{prop:pointwiseChangeActions}
  Every change action $\cstruct{B}{\changes{B}}{\cplus}$ such that the change action
  $\cstruct{\changes{B}}{\changes{B}}{\splus}$ is complete and $\cplus$ is differentiable with
  respect to its first argument is pointwise.

  In particular, every change action where $\cplus$ is a group action is pointwise.
\end{prop}
\ifproofs
  See \cref{prf:pointwiseChangeActions}.
\fi

\begin{prop}[name=Complete implies pointwise, restate=pointwiseComplete]
  \label{prop:pointwiseComplete}
   Every complete change action $\cstr{B}$ is pointwise.
\end{prop}
\ifproofs
  See \cref{prf:pointwiseComplete}.
\fi

Furthermore, since the products and exponentials of complete change actions are complete,
the full subcategory of $\cat{CAct}$ formed by complete change actions is a Cartesian
closed category of pointwise change actions.

Derivatives of the evaluation map, irrespective of a particular choice of change action on the
exponential objects, give us a way to do incremental evaluation of function application:
\begin{prop}[Incremental application]
\label{prop:incrementalization}
  Let $f: \cstr{A} \rightarrow \cstr{B}$, $a \in A$, $\change{a} \in
  \changes{A}$, $\change{f} \in \changes{(A \rightarrow B)}$, and let
  $\derive{\ev}$ be a derivative of the evaluation map.

  Then
  \begin{displaymath}
    (f \cplus \change{f})(a \cplus \change{a}) = f(a) \cplus \derive{\ev}((f, a), (\change{f}, \change{a}))
  \end{displaymath}
\end{prop}

Conveniently, \cref{def:pointwiseChanges} gives us a way to compute explicit derivatives for
the evaluation map on pointwise change actions:

\begin{prop}[Derivatives of the evaluation map]
\label{prop:evDerivatives}
  Suppose $\cstr{B}$ is a pointwise change action. Let
  $f: \cstr{A} \rightarrow \cstr{B}$,
  $a \in A$, $\change{a} \in \changes{A}$,
  $\change{f} \in \changes{(A \rightarrow B)}$.

  Then, by taking a derivative of $f$ we obtain the following derivative for the evaluation map:
  \begin{displaymath}
    \derive{\ev}_1((f, a), (\change{f}, \change{a})) \defeq \derive{f}(a, \change{a}) \splus \change{f}(a \cplus \change{a})
  \end{displaymath}

  Alternatively, by taking a derivative of $f \cplus \change{f}$ we can obtain another derivative
  for the evaluation map:
  \begin{displaymath}
    \derive{\ev}_2((f, a), (\change{f}, \change{a})) \defeq \change{f}(a) \splus \derive{(f \cplus \change{f})}(a, \change{a})
  \end{displaymath}
\end{prop}

This is important for \cref{sec:fixpoints}, where we will need derivatives of
the evaluation map. In practice, this means that we will only be able to compute
these results when we have pointwise change actions, since we know how to
compute derivatives of the evaluation map for them.

\section{Change actions over ordered structures}
\label{sec:moreStructures}

We are aiming to work over Boolean algebras with fixpoints, which is where we
will interpret Datalog. We will work up to that gradually, adding power as we
progress from posets, to directed-complete partial orders, and
finally to Boolean algebras.

\subsection{Posets}

Ordered sets give us an interestingly constrained class of functions: monotone
functions. We can define \emph{ordered} change actions, which are well-behaved
with respect to monotonicity.\footnote{If we were giving a presentation that was
generic in the base category, then this would simply be the definition of being
a change action over posets.}

\begin{defn}[Ordered change actions]
  A change action $\cstr{A}$ is \emph{ordered} if
  \begin{itemize}
    \item $A$ and $\changes{A}$ are posets.
    \item $\cplus$ and $\splus$ are monotone in both arguments.
  \end{itemize}
\end{defn}

In fact, in all cases where the base set is a poset, we can always pick an order on the change
set which makes $\cplus$ monotone in its second argument.

\begin{defn}[Change preorder]
  $\change{a} \changeOrder \change{b}$ iff for all $a \in A$ it is the case that $a \cplus \change{a} \leq a \cplus \change{b}$.
\end{defn}

Ordered change actions are useful because they let us give a sensible pointwise preorder
to our derivatives.

\begin{thm}[Sandwich lemma]
  \label{thm:sandwich}
  Let $\cstr{A}$ be an change action, $\cstr{B}$ be an ordered change action,
  $f: \cstr{A} \rightarrow \cstr{B}$, and $\supderive{f}$ and $\subderive{f}$ be
  derivatives for $f$ such that

  \begin{displaymath}
    \supderive{f} \changeOrder g \changeOrder \subderive{f}
  \end{displaymath}

  Then $g$ is a derivative for $f$.
\end{thm}

If we have unique minimal and maximal derivatives then this gives us a full
characterisation of all the derivatives for a function.

\begin{thm}[Characterization of derivatives]
\label{thm:derivativeCharacterization}
  Let $\cstr{A}$ and $\cstr{B}$ be change actions, let
  $f: \cstr{A} \rightarrow \cstr{B}$ be a function, and let $\subderiveM{f}$ and
  $\supderiveM{f}$ be unique minimal and maximal derivatives of $f$, respectively.
  Then the derivatives of $f$ are precisely
  the functions $\derive{f}$ such that
  \begin{displaymath}
    \subderiveM{f} \changeOrder \derive{f} \changeOrder \supderiveM{f}
  \end{displaymath}
\end{thm}
\ifproofs
\begin{proof}
  Follows easily from \cref{thm:sandwich} and minimality/maximality.
\end{proof}
\fi

This theorem gives us leeway when trying to pick a derivative: we can pick out the
bounds, and that tells us how much ``wiggle room'' we have. This is helpful
because some of the intermediary functions may be much easier to compute than
others, as we will see in \cref{sec:datalogDifferentiability}.

One way to get minimal and maximal derivatives is from minimal and maximal minus
operators (if we are in a complete change action).

\begin{defn}[Minus ordering]
  $\cminus_1 \minusOrder \cminus_2$ iff for all $a,b \in A$, $a \cminus_1 b
  \changeOrder a \cminus_2 b$.
\end{defn}

This implies an ordering on the corresponding derivatives: if $\cminus_1 \minusOrder \cminus_2$ then
$\derive{f}_{\cminus_1} \changeOrder \derive{f}_{\cminus_2}$.

Which gives us a correspondence between maximal minus operators and maximal derivatives.

\begin{prop}
  \label{prop:maximalMinusDerivatives}
  If $\cminus$ is a minimal (maximal) minus operator with respect to
  $\minusOrder$, then $\derive{f}_\cminus$ is a minimal (maximal) derivative.
\end{prop}

\subsection{Directed-complete partial orders}

Directed-complete partial orders with least elements (dcpos) give us the well known notion of
\emph{Scott-continuity}. We can define \emph{continuous} change actions,
which are well-behaved with respect to continuity.

\begin{defn}[Continuous change actions]
  A change action $\cstr{A}$ is \emph{continuous} if
  \begin{itemize}
    \item $A$ and $\changes{A}$ are dcpos.
    \item $\cplus$ and $\splus$ are Scott-continuous in both arguments.
  \end{itemize}
\end{defn}

We also have the following lemma (which is just a re-statement of a well-known
property of Scott-continuous functions, see e.g. \cite[Lemma~3.2.6]{abramsky1994domain}):

\begin{prop}[Distributivity of limits across arguments]
  \label{prop:distributivityLimit}
  A function $f : A \times B \rightarrow C$ is continuous iff it is continuous in each variable separately.
\end{prop}

A direct corollary of this property is the following result, reminiscent of a well-known theorem of calculus:

\begin{corollary}[Continuity of differentiation]
  \label{cor:diffContinuous}
  Let $\cstr{A}$, $\cstr{B}$ be change actions, with $\cstr{B}$ continuous and let $\{f_i\}$ and $\{\derive{f_i}\}$ be
  $I$-indexed directed families of functions in $A \rightarrow B$ and $A \times \changes{A} \rightarrow \changes{B}$.

  Then, if for every $i \in I$ it is the case that $\derive{f_i}$ is a derivative of $f_i$, then $\sqcup_{i \in I} \{ \derive{f_i} \}$ is
  a derivative of $\sqcup_{i \in I} \{ f_i \}$.
\end{corollary}
\ifproofs
\begin{proof}
  It suffices to apply $\cplus$ and \cref{prop:distributivityLimit} to the directed families $\{ f_i(a) \}$ and
  $\{ \derive{f_i}(a, \change{a}) \}$.
\end{proof}
\fi

We also state the following additional fixpoint lemma. This is a specialization of
Becik's Theorem \autocite[][section 10.1]{winskel1993formal}, but it has a straightforward direct proof.

\begin{prop}[name=Factoring of fixpoints, restate=factoringFixpoints]
  \label{prop:factoringFixpoints}
  Let $A$ and $B$ be dcpos, $f : A \rightarrow A$ and $g: A \times B \rightarrow B$ be continuous, and let
  \begin{displaymath}
    h(a, b) \defeq (f(a), g(a, b))
  \end{displaymath}
  Then
  \begin{displaymath}
    \lfp(h) = (\lfp(f), \lfp(g(\lfp(f))))
  \end{displaymath}
\end{prop}
\ifproofs
\begin{proof}
  See \cref{prf:factoringFixpoints}.
\end{proof}
\fi

\subsection{Boolean algebras}
\label{sec:booleanAlgebras}

Complete Boolean algebras have a complete, continuous change action.

\begin{prop}[name=Boolean algebra change actions, restate=lsuperpose]
 Let $L$ be a complete Boolean algebra. Define
  \begin{displaymath}
    \cstr{L}_\superpose \defeq \cstruct{L}{L \times L}{\twist}
  \end{displaymath}
  where
  \begin{displaymath}
    a \twist (p, q) \defeq (a \vee p) \wedge \neg q
  \end{displaymath}
  and the monoid operator is
  \begin{displaymath}
    (p, q) \splus (r, s) \defeq ((p \wedge \neg q) \vee r, (q \wedge \neg r) \vee s)
  \end{displaymath}

  Then $\cstr{L}_\superpose$ is a complete (and hence pointwise), continuous change action on $L$.
\end{prop}
\ifproofs
\begin{proof}
  See \cref{prf:lsuperpose}.
\end{proof}
\fi

We can think of $\cstr{L}_\superpose$ as tracking changes as pairs of ``upwards'' and
``downwards'' changes, where the monoid action simply applies both.\footnote{We
  can, in fact, make it precise that $\cstr{L}_\superpose$ is an ``upwards''
  and ``downwards'' change action glued together, but here it is simpler to
  just go directly to the useful change action.}  

Boolean algebras also have concrete definitions for unique maximal and minimal minus
operators, giving them the usual preorder based on implication.\footnote{The change
set is, as usual, given the change preorder, which in this case corresponds to
the natural order on $L \times L^{\textrm{op}}$.}

\begin{prop}
  Let $L$ be a Boolean algebra. Then
  \begin{align*}
    a \cminus_\bot b &= (a \wedge \neg b, b)\\
    a \cminus_\top b &= (a, b \wedge \neg a)
  \end{align*}

  define unique minimal and maximal minus operators.
\end{prop}

In particular, \cref{thm:derivativeCharacterization} along with
\cref{prop:maximalMinusDerivatives} give us bounds for
all the derivatives on Boolean algebras:

\begin{corollary}
\label{cor:booleanCharacterization}
  Let $L$ be a Boolean algebra with the $\cstr{L}_\superpose$ change action, $A$ be
  a change action, and $f: A \rightarrow
  L$ a function. Then the derivatives of $f$ are precisely those functions
  $\derive{f}$ such that
  \begin{displaymath}
    f(a \cplus \change{a}) \cminus_\bot f(a)
    \changeOrder
    \derive{f}(a, \change{a})
    \changeOrder
    f(a \cplus \change{a}) \cminus_\top f(a)
  \end{displaymath}
\end{corollary}

This makes \cref{thm:derivativeCharacterization} actually usable in practice, as
we have concrete definitions for our bounds (which, again, we will make use of in \cref{sec:datalogDifferentiability}).

\section{Fixpoints}
\label{sec:fixpoints}

Fixpoints appear frequently in the semantics of languages with recursion. If we
can give a generic account of how to compute fixpoints using change actions,
then this gives us a compositional way of extending a change semantics for a
language to handle the addition of recursion. We will later do this to handle
recursion in Datalog (\cref{sec:datalogIncr}).

\subsection{Iteration functions}

Over directed-complete partial orders we can define a least fixpoint operator $\lfp$ in terms of the
iteration function $\iter$:
\begin{align*}
  &\lfp : (A \rightarrow A) \rightarrow A\\
  &\lfp \defeq \sqcup_{n \in \NN} \{ \iter_n \}\\
  &\iter : (A \rightarrow A) \times \NN \rightarrow A\\
  &\iter(f, n) \defeq f^n(\bot)
\end{align*}

The iteration function is going to be the basis for everything in this section:
we can take the partial derivative with respect to $n$, and this will give us a way to get
to the next iteration incrementally; and we can compute the partial derivative
with respect to $f$, and this will give us a way to get from iterating $f$ to iterating $f
\cplus \change{f}$.\footnote{The sharp-eyed reader may have noticed that we
  could also abstract out the base point (which we have just specified as
  $\bot$) and differentiate with respect to that.}

To avoid confusion, we shall write $\iter_f$ when we are holding $f$ constant,
and $\iter_n$ when we are holding $n$ constant.

\subsection{Incremental computation of fixpoints}

The following theorems provide a
generalization of semi-naive evaluation to any differentiable function over a
continuous change action. We will want to apply this to the semantics of
Datalog, for which we will need to differentiate its semantics. We will see the
details of how to do this in \cref{sec:datalogDifferentiability}.

Since we are trying to incrementalize the iterative step, we start by taking the partial
derivative of $\iter$ with respect to $n$.

\begin{prop}[name=Derivative of the iteration map with respect to $n$, restate=iterDerivativesN]
  \label{prop:iterDerivativesN}
  Let $\cstr{A}$ be a complete change action. Then $\iter_f$ is differentiable, and a derivative is given by:
  \begin{align*}
    &\derive{\iter_f}: \NN \times \changes{\NN} \rightarrow \changes{A}\\
    &\derive{\iter_f}(0, m) \defeq \iter_f(m) \cminus \bot\\
    &\derive{\iter_f}(n+1, m) \defeq \derive{f}(\iter_f(n), \derive{\iter_f}(n, m))
  \end{align*}
\end{prop}
\ifproofs
\begin{proof}
  See \cref{prf:iterDerivativesN}.
\end{proof}
\fi

We can then compute $\derive{\iter_f(n)}$ along with $\iter_f(n)$ via mutual recursion.
We want to do this by computing a fixpoint, so we can rewrite it as a recurrence
relation:
\begin{align*}
  &\nextiter_f : (A, \changes{A}) \rightarrow (A, \changes{A})\\
  &\nextiter_f (\bot, \bot) \defeq (\bot, f(\bot) \cminus \bot)\\
  &\nextiter_f (a, \change{a}) \defeq (a \cplus \change{a}, \derive{f}(a, \change{a}))
\end{align*}
Which has the property that
\begin{align*}
  &\nextiter_f^n (\bot, \bot) = (\iter_f(n), \derive{\iter_f}(n, 1))
\end{align*}

\begin{thm}[name=Incremental computation of least fixpoints, restate=fixpointIter]
\label{thm:fixpointIter}
  Let $\cstr{A}$ be a continuous change action, $f: \cstr{A} \rightarrow
  \cstr{A}$ be continuous and differentiable.

  Then $\lfp(f) = \sqcup_{n \in \NN}(\pi_1(\nextiter_f^n(\bot)))$.
\end{thm}
\ifproofs
\begin{proof}
  See \cref{prf:fixpointIter}.
\end{proof}
\fi

This gives us a way to compute our fixpoint incrementally, by adding successive
changes to our accumulator until we reach fixpoint.

Note that we have \emph{not} taken the fixpoint of $\nextiter_f$, since it is
not continuous. 

\subsection{Derivatives of fixpoints}
\label{sec:fixpointDerivatives}

The previous section has shown us how to use derivatives to compute fixpoints
more efficiently, but we might also want to take the derivative of the fixpoint
operator itself. The typical case for this will be where we have some fixpoint
\begin{displaymath}
  \fixpoint (\lambda X . F(E, X))
\end{displaymath}
and we now wish to apply a change to $E$ and compute
\begin{displaymath}
  \fixpoint (\lambda X . F(E \cplus \change{E}, X))
\end{displaymath}

This amounts to applying a change to the \emph{function} whose fixpoint we are taking.

In Datalog this would allow us to update a recursively defined relation given an
update to one of its non-recursive dependencies, or the base extensional database.
For example, we might want to take the transitive closure relation
and update it by changing the edge relation $e$.

However, this requires us to have a derivative for the fixpoint operator
$\fixpoint$ with respect to the function which it takes the fixpoint of.

\begin{defn}
  Let $\cstr{A}$ be a change action, $\fixpoint_A$ a fixpoint operator, and
  $\derive{\ev}$ be a derivative of the evaluation map.
  
  Then we define
  \begin{align*}
    &\adjust : (A \rightarrow A) \times \changes{(A \rightarrow A)} \rightarrow (\changes{A} \rightarrow \changes{A})\\
    &\adjust(f, \change{f}) \defeq \lambda\ \change{a} . \derive{\ev}((f,
    \fixpoint_A(f), (\change{f}, \change{a})))\\
    &\derive{\fixpoint_A} : (A \rightarrow A) \times \changes{(A \rightarrow A)} \rightarrow \changes{A}\\
    &\derive{\fixpoint_A}(f, \change{f}) \defeq \fixpoint_{\changes{A}}(\adjust(f, \change{f}))
  \end{align*}
\end{defn}

\begin{thm}[name=Pseudo-derivatives of fixpoints, restate=fixpointPseudoDerivatives]
\label{thm:fixpointPseudoDerivatives}
  Let
  \begin{itemize}
    \item $\cstr{A}$ be a change action
    \item $\fixpoint : (\cstr{A} \rightarrow \cstr{A}) \rightarrow \cstr{A}$ be a fixpoint operator
    \item $f: \cstr{A} \rightarrow \cstr{A}$ be a differentiable function
    \item $\change{f} \in \changes{(A \rightarrow A)}$ be a function change 
    \item $\derive{\ev}$ be a derivative of the evaluation map
  \end{itemize}

  Then a change $\change{w} \in \Delta A$ satisfies
  the equation:
  \begin{equation}\label{eqn:fixcondition}
    \change{w} = \adjust(f, \change{f})(\change{w})
  \end{equation}
  if and only if $\fixpoint(f) \cplus \change{w}$ is a fixpoint of $f \cplus \change{f}$.

  In particular, $\fixpoint(f) \cplus \derive{\fixpoint}(f, \change{f})$ is a fixpoint
  of $f \cplus \change{f}$.
\end{thm}
\ifproofs
\begin{proof}
  See \cref{prf:fixpointPseudoDerivatives}.
\end{proof}
\fi

This is not enough to give us a true derivative, because we have only shown
that $\fixpoint(f) \cplus \derive{\fixpoint}(f, \change{f})$ computes \emph{a} fixpoint, not necessarily
the same one computed by $\fixpoint{(f \cplus \change{f})}$.

However, if we restrict ourselves to directed-complete partial orders, least
fixpoints, and continuous change actions, then $\derive{\lfp}$ \emph{is} a
derivative of $\lfp$. This is not too onerous a restriction, since this is
the setting in which we normally compute fixpoints anyway.

Since $\lfp$ is characterized as the limit of a chain of functions,
\cref{cor:diffContinuous} suggests a way to compute its derivative. If we can find a derivative
$\derive{\iter_n}$ of each iteration map 
such that the resulting set $\{ \derive{\iter_n} \}$ is directed, then $\sqcup_{n \in \NN}\derive{\iter_n}$ will be a derivative of $\lfp$.

These correspond to the other derivative of $\iter$ - this time with respect to
$f$. While we are differentiating with respect to $f$, we are still going to
need to define our derivatives inductively in terms of $n$.

\begin{prop}[name=Derivative of the iteration map with respect to $f$, restate=iterDerivativesF]
  \label{prop:iterDerivativesF}
  $\iter_n$ is differentiable and a derivative is given by:
  \begin{align*}
    &\derive{\iter_n} : (A \rightarrow A) \times \changes{(A \rightarrow A)} \rightarrow \changes{A}\\
    &\derive{\iter_0} (f, \change{f}) \defeq \bot_{\changes{A}}\\
    &\derive{\iter_{n+1}} (f, \change{f}) \defeq \derive{\ev}((f, \iter_n(f)), (\change{f}, \derive{\iter_n}(f, \change{f})))
  \end{align*}
\end{prop}
\ifproofs
\begin{proof}
  See \cref{prf:iterDerivativesF}.
\end{proof}
\fi

As before, we can now compute $\derive{\iter_n}$ together with $\iter_n$ by
mutual recursion. 
\begin{align*}
  &\nextiter_{f, \change{f}} : (A, \changes{A}) \rightarrow (A, \changes{A})\\
  &\nextiter_{f, \change{f}} (a, \change{a}) \defeq (f(a), \derive{\ev}((f, a), (\change{f}, \change{a})))
\end{align*}
Which has the property that
\begin{align*}
  &\nextiter_{f, \change{f}}^n (\bot, \bot) = (\iter_n(f), \derive{\iter_n}(f, \change{f}))
\end{align*}

In fact, the recursion here is not \emph{mutual}: $\iter_n$ does
not depend on $\derive{\iter_n}$. However, writing it in this way makes it
amenable to computation by fixpoint, and we will in fact be able to avoid the recomputation of $\iter_n$.

\begin{thm}[name=Derivatives of least fixpoint operators, restate=leastFixpointDerivatives]
  \label{thm:leastFixpointDerivatives}
  Let
  \begin{itemize}
    \item $\cstr{A}$ be a continuous change action
    \item $f : \cstr{A} \rightarrow \cstr{A}$ be a continuous, differentiable function
    \item $\change{f} \in A \rightarrow \changes{A}$ be a function change
    \item $\derive{\ev}$ be a derivative of the evaluation map, continuous with
      respect to $a$ and $\change{a}$.
  \end{itemize}
  Then $\derive{\lfp}$ is a derivative of $\lfp$.
\end{thm}
\ifproofs
\begin{proof}
  See \cref{prf:leastFixpointDerivatives}.
\end{proof}
\fi

Computing the derivative still requires computing a fixpoint (over the change
lattice), but this may still be significantly less expensive than the
alternative ``update strategy'': throw everything away and start
again (which will itself require a fixpoint computation).

\section{Datalog}
\label{sec:datalog}

Datalog is a well-known simple logic programming language \autocite[See][part D]{abiteboul1995foundations}.
It is also a textbook example of the need for incremental computation, since as we have seen
(\cref{sec:intro}) the naive computation of its semantics can be very expensive.

The aim of this section is to argue that by viewing the computation
of Datalog semantics as composed of differentiable functions we can
bring the machinery that we've developed so far to bear, and give a flexible
account of incremental evaluation. 

Firstly, however, we must show that we can see the semantics of Datalog in terms
of elements that we know how to handle: Boolean algebras and fixpoints.

\subsection{Denotational semantics}

Datalog is usually given a logical semantics wherein we look for models that
satisfy the program. We will instead give a simple denotational semantics that treats a Datalog
program as denoting a family of relations.

We will adopt an equivalent of the usual closed-world assumption to give a
denotation to give a denotation to negation.

\begin{defn}[Closed-world assumption and negation]
  There exists a universal relation $\universalRel$.

  Negation on relations is defined as
  \begin{displaymath}
    \neg R \defeq \universalRel \setminus R
  \end{displaymath}
\end{defn}

This makes $\Rel$, the set of all relations over $\universalRel$, into a
complete Boolean
algebra.\footnote{Technically we need to use the set of relations of a given
  arity. In practice it is easy to make the arities match up, and we will gloss
  over them for clarity.}

\begin{defn}[Formula semantics]
  A Datalog formula $T$ denotes a function from its free relation variables to
  $\Rel$, $\denote{\_} : \Formula \rightarrow \Rel^n \rightarrow \Rel$

  $\denote{t}$ takes an argument $\overline{X} : \Rel^n$. However, this is
  passed down unchanged through all recursive calls, so we will leave it
  implicit except where it is used.

  \begin{align*}
    \denote{R_j} &\defeq X_j \tag{where $X_j$ is the $j$th element of $\overline{X}$}\\
    \denote{T \wedge U} &\defeq \denote{T} \cap \denote{U}\\
    \denote{T \vee U} &\defeq \denote{T} \cup \denote{U}\\
    \denote{\exists x. T} &\defeq \pi_x(\denote{T}) \tag{where $\pi_x(R)$ is the projection onto all columns except $x$}\\
    \denote{\neg T} &\defeq \neg \denote{T}\\
    \denote{x=y} &\defeq \Delta(x, y) \tag{where $\Delta$ is the diagonal relation}
  \end{align*}
\end{defn}

Since $\Rel$ is a complete Boolean algebra, and so is $\Rel^n$, $\denote{t}$ is
a function between complete Boolean algebras.

\begin{defn}[Immediate consequence operator]
  Given a program $\mathcal{P} = \overline{P}$, the immediate consequence operator $\consq: \Rel^n \rightarrow \Rel^n$ is defined as follows:
  \begin{displaymath}
    \consq(\overline{R}) = \overline{\denote{P_j}(\overline{R})} \tag{where $P_j$ is the $j$th component of $\overline{P}$}
  \end{displaymath}
\end{defn}

That is, given a value for the program, we pass in all the relations
to the denotation of each predicate, to get a new product of relations.

\begin{defn}[Program semantics]
  The semantics of a program $\mathcal{P}$ is defined to be
  \begin{displaymath}
    \denote{\mathcal{P}} \defeq \lfp_{\Rel^n}(\consq)
  \end{displaymath}
  and may be calculated by iterative application of $\consq$ to $\bot$ until
  fixpoint is reached.
\end{defn}

Whether or not this program semantics exists will depend on whether the fixpoint
exists. Typically this is ensured by constraining the program such that $\consq$
is monotone (or, in the context of a dcpo, continuous). We can be agnostic
about this when applying \cref{thm:fixpointIter}, but it will be a requirement to
apply \cref{thm:leastFixpointDerivatives}.

\subsection{Differentiability of Datalog semantics}
\label{sec:datalogDifferentiability}

In order to apply the results in \cref{sec:fixpoints} to Datalog, we need the semantics $\denote{\_}$ to
be differentiable. Since $\denote{\_}$ is a function on complete Boolean algebras, and we know
that the $\cstr{L}_\superpose$ change action is complete, in fact we
\emph{do} know that $\denote{\_}$ must be differentiable.

However, this does not mean that we have a \emph{good} derivative for
$\denote{\_}$. The derivative that we know we have for complete change actions
is quite bad:
\begin{displaymath}
  \derive{f}(a, \change{a}) = f(a \cplus \change{a}) \cminus f(a)
\end{displaymath}
Naively computed, this requires \emph{more} work than evaluating $f(a \cplus \change{a})$ directly!

However, \cref{cor:booleanCharacterization} gives us a range of derivatives to
choose from, and we can optimize within that range to find one that satisfies
our constraints.

In the case of Datalog, the change ordering on the change action also
corresponds to the size of the derivative as a pair of relations. The minimal
derivative contains precisely the elements that are newly added or removed,
whereas the maximal derivative contains all the elements that have \emph{ever}
been added or removed but not re-added. This means that \cref{cor:booleanCharacterization} allows
us to \emph{approximate} the most precise derivative while still being
guaranteed that the result is sound.\footnote{The idea of using an approximation
to the precise derivative, and a soundness condition, appears in \textcite{bancilhon1986amateur}.}

There is also the question of how to compute the derivative. Fortunately, the
maximal and minimal minus operators are actually representable as pairs of formulae
in Datalog, and so we can compute the derivative via a pair of formualae that
satisfy those bounds, allowing us to reuse our machinery for evaluating Datalog
formulae.\footnote{Indeed, if this process is occurring in an optimizing compiler,
  the derivative formulae can themselves be optimized. This is very 
  beneficial in practice, since the initial formulae may be quite complex.}

This does give us additional constraints that the derivative formulae must satisfy:
for example, we need to be able to evaluate them; and we may wish to pick formulae that will be easy or cheap
for our evaluation engine to compute, even if the results are larger.

The upshot of these considerations is that the optimal choice of derivatives is likely
to be quite dependent on the precise variant of Datalog being evaluated, and the
specifics of the evaluation engine. Here is one possibility.\footnote{These are
  the rules actually in use at Semmle. We arrived at them by starting with the
  minimal derivative and then simplifying and weakening it while preserving the
  bound given by the maximal derivative.}

\newcommand{\bothdiff}{\diamond}
\newcommand{\bothchanges}{\rho}
\begin{thm}[Concrete Datalog formula derivatives]
\label{thm:concreteDatalog}
  We give two mutually recursive definitions,
  $\updiff: \Formula \rightarrow \changes{\Rel^n} \rightarrow \changes{\Rel}$ and
  $\downdiff: \Formula \rightarrow \changes{\Rel^n} \rightarrow \changes{\Rel}$, such
  that $\updiff(t) \times \downdiff(t)$ is a derivative for the semantics of a Datalog formula $t$.

  $\updiff$ and $\downdiff$ define functions which take an argument
  $\bothchanges: \changes{\Rel^n}$ . However, this argument is passed down
  unchanged through all recursive calls until it is used, so we shall leave it
  implicit until it is used.

  Additionally, let
  \begin{displaymath}
    \diamond X \defeq X \twist (\updiff X, \downdiff X)
  \end{displaymath}
  in the following.

  \begin{align*}
  \updiff(\bot) &\defeq \bot\\
  \updiff(\top) &\defeq \bot\\
  \updiff(R_j) &\defeq \updiff R_j \tag{where $\updiff R_j$ is the first component of the $j$th element of $\bothchanges$}\\
  \updiff(T\vee U) &\defeq \updiff T \vee \updiff U\\
  \updiff(T\wedge U) &\defeq (\updiff T\wedge \bothdiff U)
                           \vee
                           (\updiff U \wedge \bothdiff T)\\
  \updiff(\neg T) &\defeq \downdiff T\\
  \updiff(\exists x.T) &\defeq \exists x.\updiff T\\
    \\
  \downdiff\bot &\defeq \bot\\
  \downdiff\top &\defeq \bot\\
  \downdiff R &\defeq \downdiff R_j \tag{where $\downdiff R_j$ is the second component of the $j$th element of $\bothchanges$}\\
  \downdiff(T\vee U) &\defeq (\downdiff T \wedge \neg \bothdiff U)
                           \vee
                           (\downdiff U \wedge \neg \bothdiff T)\\
  \downdiff(T\wedge U) &\defeq (\downdiff T\wedge U) \vee (T \wedge \downdiff U)\\
  \downdiff(\neg T) &\defeq \updiff T\\
  \downdiff(\exists x.T) &\defeq \exists x.\downdiff T \wedge \neg \exists x.\bothdiff T
  \end{align*}
\end{thm}
\ifproofs
\begin{proof}
  Straightforward structural induction.
\end{proof}
\fi

There is a symmetry between the cases for $\wedge$ and $\vee$ between $\updiff$
and $\downdiff$, but the cases for $\exists$ look quite different.
This is because we have chosen a dialect of Datalog without a primitive universal quantifier.
If we did have one, its cases would be dual to those for $\exists$, namely:
\begin{align*}
\updiff(\forall x.T) &= \exists x. \updiff T \wedge \forall x. \bothdiff T\\
\downdiff(\forall x.T) &= \forall x. \downdiff T
\end{align*}

We can now give a derivative for our $treeP$ program!

Here is the upwards difference:
\begin{align*}
  \updiff treeP(x) \leftarrow & p(x)\\
    &\wedge
    \exists y. (
      child(x, y)
      \wedge
      \updiff treeP(y)
    )\\
    &\wedge 
    \neg \exists y. (
      child(x, y)
      \wedge
      \neg \bothdiff treeP(y)
    )
\end{align*}

This is still not especially easy to compute. The third conjunct amounts to
recomputing the whole of the recursive part. The second conjunct gives us a
guard: we only need to do the work if there is \emph{some} change in the body of
our existential. This shows that our derivatives aren't a panacea: it is simply \emph{hard} to compute
downwards differences for $\exists$ (and, equivalently, upwards differences for
$\forall$). However, this still allows us to avoid re-evaluating the whole
formula in many cases, and the inefficiency is local to this subformula.

\subsection{Incremental evaluation of Datalog}
\label{sec:datalogIncr}

We can easily extend a derivative for the formula semantics to a derivative for
the immediate consequence operator $\consq$.

\begin{corollary}
\label{corollary:consqDiff}
  $\consq$ is differentiable.
\end{corollary}

Putting this together with the results from \cref{sec:fixpoints}, we get our two
big results.

\begin{thm}[Incremental evaluation of Datalog semantics]
\label{thm:diffEval}
  Datalog program semantics can be evaluated incrementally.
\end{thm}
\ifproofs
\begin{proof}
  Corollary of \cref{thm:fixpointIter} and \cref{corollary:consqDiff}.
\end{proof}
\fi

\begin{thm}[Incremental update of Datalog semantics]
\label{thm:diffUpdate}
  Datalog program semantics can be incrementally updated with changes to
  extensional relations.
\end{thm}
\ifproofs
\begin{proof}
  Corollary of \cref{thm:leastFixpointDerivatives} and \cref{corollary:consqDiff}.
\end{proof}
\fi

\subsection{Extensions to Datalog}
\label{sec:extensions}

Our formulation of Datalog formula semantics and incremental evaluation is 
generic and modular, so it is easy to extend the language with new
formula constructs.

A new formula construct must have:
\begin{itemize}
  \item An interpretation as a function on its free relation variables.
  \item An implementation of $\updiff$ and $\downdiff$.
\end{itemize}

These are easy conditions to satisfy. The former is needed for the construct to even
make sense, and because we are using the complete change action
$\cstr{L}_\superpose$, the latter can \emph{always} be satisfied by using the maximal or
minimal derivative. This justifies our claim that we can support
\emph{arbitrary} additional formula constructs: although the maximal and minimal
derivatives are likely to be impractical, having them
available as options means that we can still use the overall process.
Even if we have a formula construct which we cannot find a good derivative for, we only lose incremental
evaluation for those subformulae, and not for anything else.

This is important in practice for Semmle's variant of Datalog. For example, the
upwards derivative for aggregates is:

\begin{align*}
  \updiff{\textrm{agg}(vs | T | U)} \defeq \exists vs. (T \wedge \updiff{U}) \wedge \textrm{agg}(vs | T | U)
\end{align*}

While we can't give a precise derivative, similarly to the earlier example of the
upwards difference of a universal quantifier, we can still provide a guard that
prevents us from having to re-evaluate in all cases. 

\section{Related work}

\subsection{Change actions and incremental computation}

\subsubsection{Change structures}
\label{sec:relatedChangeStructures}

The seminal paper in this area is \textcite{cai2014changes}. We deviate from
that excellent paper in three regards: the
inclusion of minus operators, the nature of function changes, and the use of
dependent types.

We have omitted minus operators from our definition because
there are many interesting change actions that are not complete and so cannot
have a minus operator. Where we can find a change structure with a minus operator, often we are
forced to use unwieldy representations for change sets, and
\citeauthor{cai2014changes} cite this as their reason for using a dependently
typed type of changes. For example, the monoidal change actions on sets and lists are clearly
useful for incremental computation on streams, yet they do not admit minus operators \textemdash instead, one would
be forced to work with e.g. multisets admitting negative arities, as \citeauthor{cai2014changes} do.

Our function changes (when well behaved) correspond to what \citeauthor{cai2014changes} call
\emph{pointwise differences} \autocite[See][section 2.2]{cai2014changes}. As they point out, you can reconstruct their
function changes from pointwise changes and derivatives, so the two formulations
are equivalent. However, our function changes correspond to the
exponentials that we get from the categorical equivalence with $\cat{PreOrd}$,
and (when we can use them) pointwise differences are easier to work with.

The equivalence of our presentations means that our work should be compatible
with ILC. The rules we give in \cref{sec:datalogDifferentiability} are more or
less a ``change semantics'' for Datalog \autocite[See][section
3.5]{cai2014changes}. 

\subsubsection{S-acts}

S-acts and their categorical structure have received a fair amount of attention
over the years (\textcite{kilp2000monoids} is a good
overview). However, there is a key difference between our $\cat{CAct}$ and the category of
S-acts $\cat{SAct}$: the objects of $\cat{SAct}$ all maintain the same monoid
structure, whereas we are interested in changing both the base set \emph{and} the structure of the act.

There are similarities: if we compare the definition of an ``act-preserving''
homeomorphism in $\cat{SAct} $\autocite[See][]{kilp2000monoids} we can see that the structure is
quite similar to the definition of differentiability:
\begin{displaymath}
  f(a \splus s) = f(a) \splus s
\end{displaymath}
as opposed to
\begin{displaymath}
  f(a \cplus s) = f(a) \cplus \derive{f}(a, s)
\end{displaymath}
That is, we use $\derive{f}$ to transform the action element into the new
monoid, whereas in $\cat{SAct}$ it simply remains the same.

In fact, $\cat{SAct}$ is a subcategory of $\cat{CAct}$, where we only
consider change actions with change set $S$, and the only functions are those
whose derivative is $\lambda a. \lambda d. d$.

\subsubsection{Derivatives of fixpoints}

\textcite{arntz2017fixpoints} gives a derivative operator for fixpoints based on
the framework in \textcite{cai2014changes}. However, since we have different
notions of function changes, the result is inapplicable as
stated. In addition, we require a somewhat different set of conditions, in particular we
don't require our changes to always be increasing.

\subsection{Datalog}

\subsubsection{Incremental evaluation}

The earliest explication of semi-naive evaluation as a derivative process
appears in \textcite{bancilhon1986naive}. The idea of using an approximate derivative
and the requisite soundness condition appears as a throwaway comment in
\textcite[][section 3.2.2]{bancilhon1986amateur}, and as far as we know nobody has since
developed that approach.

As far as we know, traditional semi-naive is the state of
the art in incremental, bottom-up, Datalog evaluation, and there are no strategies that
accommodate additional language features such as parity-stratified negation and aggregates.

\subsubsection{Incremental updates}

There is existing literature on incremental updates to relational algebra
expressions. In particular \textcite{griffin1997improved} following
\textcite{qian1991incremental} shows the essential insight that it is necessary to
track both an ``upwards'' and a ``downwards'' difference, and produces a set of
rules that look quite similar to those we derive in \cref{thm:concreteDatalog}.

Where our presentation improves over \citeauthor{griffin1997improved} is mainly in
the genericity of the presentation. Our machinery works for a wider variety of
algebraic structures, and it is clear how the parts of the proof work together
to produce the result. In addition, it is easy to see how to extend the proofs
to cover additional language constructs.

There are some inessential points of difference as well: we work on Datalog,
rather than relational algebra; and we use set semantics rather than bag
semantics. This is largely a matter of convenience: Datalog is an easier
language to work with, and set semantics allows a much wider range of valid
simplifications. However, all the same machinery applies to relational algebra
with bag semantics, it is simply necessary to produce a valid version of \cref{thm:concreteDatalog}.

We also solve the problem of updating \emph{recursive} expressions. As far as we
know, this is unsolved in general. Most of the attempts to solve it have
focussed on Datalog rather than relational algebra, since Datalog is designed to
make heavy use of recursion.

Several approaches
\autocites{gupta1993maintaining}{harrison1992maintenance}
make use of a common tactic: one can get to the new fixed
point by starting from \emph{any} point below it, and then iterating the
semantics again to fixpoint. The approach, then, is to find a way to delete as
few tuples as possible to get below the new fixpoint, and then iterate again
(possibly using an incremental version of the semantics).

This is a perfectly reasonable approach, and given a good, domain-specific,
means of getting below the fixpoint, they can be quite efficient (likely more
efficient than our method). The main defect of these approaches is that they
\emph{are} domain specific, and hence inflexible with respect to changes in the
language or structure, whereas our approach is quite generic.

Other approaches \autocites{dong2000incremental}{urpi1992method} consider only
restricted subsets of Datalog, or incur other substantial constraints, and our results
are thus significantly more general.

\subsubsection{Embedding Datalog}
\label{sec:embeddingDatalog}

Datafun (\textcite{arntz2016datafun}) is a functional programming language that embeds
Datalog, allowing significant improvements in genericity, such as the use of
higher-order functions. Since we have directly defined a change action and
derivative operator for Datalog, our work could be used as a ``plugin'' in the sense
of \citeauthor{cai2014changes}, allowing Datafun to compute its internal fixpoints
incrementally, but also allowing Datafun expressions to be fully incrementally updated.

\subsection{Differential $\lambda$-calculus}

Another setting where derivatives of arbitrary higher-order programs have been studied
is the \emph{differential $\lambda$-calculus} \autocites{ehrhard2003differential}{ehrhard2017introduction}.
This is a higher-order, simply-typed
calculus which allows for computing the derivative of a function, in a similar
way to the notion of derivative in Cai's work and the present paper.

While there are clear similarities between the two systems, 
the most important difference is the properties of the derivatives themselves:
in the differential $\lambda$-calculus, derivatives are guaranteed to be linear
in their second argument, whereas in our approach derivatives do not have this restriction 
but are instead required to satisfy a strong relation to the function
that is being differentiated (see \cref{def:derivative}).

Families of denotational models for the differential $\lambda$-calculus have been
studied in depth
\autocites{bucciarelli2010categorical}{blute2010convenient}{cockett2016categorical}{kerjean2016mackey},
and the relationship between these and change actions is the subject of ongoing work.

\subsection{Higher-order automatic differentiation}

Automatic differentiation \autocite{griewank2008evaluating} is a technique that allows
for efficiently computing the derivative of arbitrary programs, with
applications in probabilistic modeling \autocite{kucukelbir2017automatic}
and machine learning \autocite{baydin2014automatic} among other areas. In recent times, this technique has been successfully
applied to higher-order languages \autocites{siskind2008nesting}{baydin2016diffsharp}.
While some approaches have been suggested \autocites{manzyuk2012simply}{kelly2016evolving}, a general
theoretical framework for this technique is still a matter of open research. 

To this purpose, some authors have proposed the incremental $\lambda$-calculus
as a foundational framework on which models of automatic differentiation can
be based \autocite{kelly2016evolving}. We believe our change actions are better suited
to this purpose than the incremental $\lambda$-calculus, since one can easily give them a
synthetic differential geometric reading (by interpreting $\cstr{A}$ as an Euclidean module and $\changes{A}$
as its corresponding spectrum, for example).

\section{Conclusions and future work}

We have presented change actions and their properties, and used them to provide novel
strategies for incrementally evaluating fixpoints and the semantics of Datalog.

Our work opens several avenues for future investigation.

Firstly, the concrete definition of change actions can be generalized in a number of
ways. A 2-categorical version (by analogy with the 2-categorical interpretation
of $\cat{PreOrd}$) would smooth over many of the technical difficulties which
the current presentation faces due to the fact that the changes produced
$\derive{f}(a, \change{a})$ can be applied at points other than $f(a)$.
A description of change actions which is generic in
the base category would also be easy to provide, which may lead to models of
the incremental $\lambda$-calculus on different base categories. 

Of those, the theory of change actions on the category of domains is of particular interest. Since
domains are used to model programming language semantics, this could
open up opportunities for incremental evaluation of many programming languages,
even those that do not fit into the model of \citeauthor{cai2014changes}'s ILC.
Our fixpoint theorems are proven over dcpos in general, which is crucial since
fixpoints are often used in domain theory to give a semantics to recursion.

Additionally, the tantalizing connection between change actions, the ILC and synthetic
differential geometry has only begun to be explored and a denotational semantics for the
ILC based on smooth spaces is the subject of ongoing research. We believe many concepts
from standard differential geometry, like gradients, vector fields, curves and flows can
be defined for general change actions, which could then have implications for higher-order
computation.

Finally, combining our concrete Datalog derivatives with a system similar to ILC
in a language such as Datafun would be an exciting demonstration of the compositional
power of this approach.

\section{Acknowledgements}

We would like to thank Semmle Ltd. for supporting this research, as well as Pavel
Avgustinov, Aditya Sharad, Max Sch\"afer, Katriel Cohn-Gordon, and Simon Peyton Jones for their
helpful comments on the manuscript.

\printbibliography

\clearpage
\appendix
\appendixpage
\section{Proofs}

\subsection{The category of change actions}

\preordEquivalence*
\begin{proof}
  \label{prf:preordEquivalence}
  On one direction, if $U$ is a preorder, it's trivial to check that $\reach (\direct (U)) = U$.

  On the other direction, we need to find a natural isomorphism between $\direct \circ \reach$
  and the identity functor. First, we note that the base set for the change action
  $\direct(\reach(A))$ is the same as the base set for $\cstr{A}$.

  We claim that the desired natural isomorphism is given by the
  the identity on the base sets. It remains to prove that the identities
  $id_A : \cstr{A} \rightarrow \direct(\reach(A))$ and
  $id_{A_{\reachOrder}} : \direct(\reach(A)) \rightarrow \cstr{A}$
  are indeed differentiable in both directions.

  In one direction, a derivative is given by
  \begin{displaymath}
    \derive{id}_A(a, \change{a}) \defeq (a, a \cplus \change{a})
  \end{displaymath}
  Conversely, let $(a, b) \in \reachOrder$. By definition of $\reachOrder$, there is some
  $\change{}_{(a, b)} \in \changes{A}$ satisfying $a \cplus \change{}_{(a,b)} = b$.
  This gives a definition for the derivative of the identity on pairs:
  \begin{displaymath}
    \derive{id}_{A_{\reachOrder}}(a, (a, b)) \defeq \change{}_{(a, b)}
  \end{displaymath}
  which can be extended freely to the whole change set
  $\reachOrder^\star$.\footnote{Note that since we picked arbitrary change
    representatives, the resulting derivative may not preserve the monoid action.}
\end{proof}

\products*
\begin{proof}
  \label{prf:products}
  Let $\cstr{Y}$ be a change action, and $f_1: \cstr{Y} \rightarrow \cstr{A}$, $f_2: \cstr{Y}
  \rightarrow \cstr{B}$ be morphisms.

  Then the product morphism in $\cat{Set}$, $\pair{f_1}{f_2}$ is the product
  morphism in $\cat{CAct}$. It can easily be
  shown that $\pair{\derive{f_1}}{\derive{f_2}}$ is a derivative of $\pair{f_1}{f_2}$,
  hence $\pair{f_1}{f_2}$ is a morphism in $\cat{SAct}$.

  Commutativity and uniqueness follow from the corresponding properties of the
  product in the $\cat{Set}$.
\end{proof}

\coproducts*
\begin{proof}
  \label{prf:coproducts}
  Let $\cstr{Y}$ be a change action, and $f_1 : \cstr{A} \rightarrow \cstr{Y}$, $f_2 : \cstr{B}
  \rightarrow \cstr{Y}$ be differentiable functions.

  As before, it suffices to prove that the universal function $[f_1, f_2]$ in $\cat{Set}$ is a differentiable
  function from $\cstruct{A + B}{\changes{A} \times \changes{B}}{\cplusvee}$ into $Y$. It's easy to see
  that the following morphism is a derivative:
  \begin{align*}
    \derive{[f_1, f_2]} (i_1 a, (\change{a}, \change{b})) &\defeq f_1'(a, \change{a})\\
    \derive{[f_1, f_2]} (i_2 b, (\change{a}, \change{b})) &\defeq f_2'(b, \change{b})
  \end{align*}
\end{proof}

\pointwiseChangeActions*
\begin{proof}
\label{prf:pointwiseChangeActions}
  Let $\cstr{B}$ be a change action, with $\cplus$ differentiable with respect
  to its first argument, and suppose the change action on $\changes{B}$
  is complete.

  First, we prove that the pointwise change action is well defined: let
  $f : \cstr{A} \rightarrow \cstr{B}$ be some differentiable function, and
  $\change{f} : A \rightarrow \changes{B}$ a pointwise change (which is
  differentiable, since $\cstr{\changes{B}}$ is complete). Then:
  \begin{align*}
    &(f \cplus_\rightarrow \change{f})(a \cplus \change{a})\\
    &= f(a \cplus \change{a}) \cplus \change{f}(a \cplus \change{a})\\
    &= (f(a) \cplus \derive{f}(a, \change{a})) \cplus (\change{f}(a) \splus \derive{\change{f}}(a, \change{a}) )\\
    &= f(a) \cplus (\change{f}(a) \splus \derive{\change{f}}(a, \change{a}))
       \cplus \derive{\cplus}((f(a), \derive{f}(a, \change{a})), \change{f}(a \cplus \change{a}))\\
    &= (f \cplus_\rightarrow \change{f})(a) \cplus (\derive{\change{f}}(a, \change{a})
       \splus \derive{\cplus}((f(a), \derive{f}(a, \change{a}), \change{f}(a, \cplus \change{a}))))
  \end{align*}
  Hence $(f \cplus_\rightarrow \change{f})$ is differentiable.
  
  The evaluation map, defined as $\ev(f, a) \defeq f(a)$ is differentiable as well:
  \begin{align*}
    &\ev((f, \change{f}), (a, \change{a}))\\
    &= (f \cplus_\rightarrow \change{f})(a \cplus \change{a}) \\
    &= f(a \cplus \change{a}) \cplus \change{f}(a \cplus \change{a})\\
    &= f(a) \cplus (f'(a, \change{a}) \splus \change{f}(a \cplus \change{a}))\\
    &= \ev(f, a) \cplus (f'(a, \change{a}) \splus \change{f}(a \cplus \change{a}))
  \end{align*}

  Now, on the one hand, suppose that $f : C \times A \rightarrow B$ is a differentiable function,
  and $\curry{f} : C \rightarrow (\exponential{\cstr{A}}{\cstr{B}})$ its corresponding curried version:
  \begin{align*}
    &\curry{f}(c \cplus \change{c})\\
    &= \lambda a . f(c \cplus \change{c}, a)\\
    &= \lambda a . f(c, a) \cplus \derive{f}((c, \change{c}), (a, 0))\\
    &= (\lambda a . f(c, a)) \cplus_\rightarrow (\lambda a . \derive{f}((c, \change{c}), (a, 0)))
  \end{align*}
  Then the function $\lambda a . \derive{f}((c, \change{c}), (a, 0))$ is a derivative for $\curry{f}$,
  thus $\curry{f}$ is differentiable. Conversely, suppose that $\curry{f}$ is differentiable.
  Then we note that, by hypothesis, $\curry{f}(a)$ is differentiable for all $f$, with derivative
  $\derive{f_a}$, and hence:
  \begin{align*}
    &f(a \cplus \change{a}, c \cplus \change{c})\\
    &= \curry{f}(a \cplus \change{a})(c \cplus \change{c})\\
    &= (\curry{f}(a) \cplus_\rightarrow \derive{\curry{f}}(a, \change{a}))(c \cplus \change{c})\\
    &= \curry{f}(a)(c \cplus \change{c}) \cplus \derive{\curry{f}}(a, \change{a})(c\cplus\change{c})\\
    &= \curry{f}(a)(c) \cplus (\derive{f_a}(c, \change{c}) \splus \derive{\curry{f}}(a, \change{a})(c\cplus\change{c}))\\
    &= f(a, c) \cplus (\derive{f_a}(c, \change{c}) \splus \derive{\curry{f}}(a, \change{a})(c\cplus\change{c}))
  \end{align*}
  Thus $f$ is differentiable as a function from $\cstr{C} \times \cstr{A}$ into $\cstr{B}$.
\end{proof}

\pointwiseComplete*
\begin{proof}
\label{prf:pointwiseComplete}
  Let $\cstr{B}$ be a complete change action.
  We prove that the pointwise change action is the exponential object $\exponential{\cstr{A}}{\cstr{B}}$.

  First, we note that the pointwise change action is trivially well defined, since
  every function into $\cstr{B}$ is differentiable. Similarly, the evaluation map
  is differentiable.

  Now consider two elements $\curry{f}$, $\curry{g}$ of $\exponential{\cstr{A}}{\cstr{B}}$. Since
  $\cstr{B}$ is complete, it is endowed with a minus operator $\cminus$, so we define
  $\cminus_\rightarrow : (\exponential{\cstr{A}}{\cstr{B}}) \times (\exponential{\cstr{A}}{\cstr{B}}) \rightarrow A \rightarrow \changes{B}$ by:
  $$ (g \cminus_\rightarrow f)(a) \defeq g(a) \cminus f(a) $$
  It's easy to check that this is indeed a minus operator and, therefore, $\exponential{\cstr{A}}{\cstr{B}}$
  is complete. Thus, trivially, any function $f : C \times A \rightarrow B$ is differentiable
  if and only if its curried version $\curry{f} : C \rightarrow (\exponential{A}{B})$ is.
\end{proof}

\subsection{Change actions over other structures}

\factoringFixpoints*
\begin{proof}
  \label{prf:factoringFixpoints}
  Let
  \begin{displaymath}
    p(b) = (\lfp f, g(\lfp(f), b))
  \end{displaymath}
  Then $h(h^i(\bot)) \leq p(p^i(\bot))$ (by simple induction), and so by continuity
  \begin{displaymath}
    \lfp(h) = \sqcup_{i \in \NN} h^i(\bot) \leq \sqcup_{i \in \NN} p^i(\bot) = \lfp(p)
  \end{displaymath}

  But $h(\lfp(p)) = \lfp(p)$, so $\lfp(h) \leq \lfp(p)$, since $\lfp(h)$ is least.

  Hence $\lfp(h) = \lfp(p) = (\lfp(f), \lfp(g(\lfp(f))))$.
\end{proof}

\lsuperpose*
\begin{proof}
  \label{prf:lsuperpose}
  We show that the monoid action property holds:
  \begin{align*}
    &a \twist \left[(p, q) \splus (r, s)\right]\\
    &= a \twist ((p \wedge \neg q) \vee r, (q \wedge \neg r) \vee s)\\
    &= \left(
      a \vee
      \left(
        \left(
          p \wedge \neg q
        \right)
        \vee r
      \right)
    \right)
    \wedge \neg
    \left(
      \left(
        q \wedge \neg r
      \right)
      \vee s
    \right)\\
    &= \left(
      \left(
        \left(
          a \vee p
        \right)
        \wedge
        \left(
          a \vee \neg q
        \right)
      \right)
      \vee r
    \right)
    \wedge
    \left(
      \neg q \vee r
    \right)
    \wedge
    \neg s
    \tag{distributing $\vee$ over $\wedge$, applying de Morgan rules}\\
    &= \left(
      \left(
        \left(
          a \vee p
        \right)
        \wedge
        \left(
          a \vee \neg q
        \right)
        \wedge
        \neg q
      \right)
      \vee r
    \right)
    \wedge
    \neg s
    \tag{un-distributing $\vee$ over $\wedge$ }\\
    &= \left(
      \left(
        \left(
          a \vee p
        \right)
        \wedge
        \neg q
      \right)
      \vee r
    \right)
    \wedge
    \neg s
    \tag{$(A \vee B) \wedge B = B$}\\
    &= a \twist (p, q) \twist (r, s)
  \end{align*}

  Completeness is easy to show.

  $L$ is a complete lattice, so certainly a dcpo. $\cstr{L}_\superpose$ is a
  dcpo with $\bigvee (p_i, q_i) \defeq (\bigvee p_i, \bigwedge q_i)$.

  Continuity of $\twist$ in its second argument:
  \begin{align*}
    &a \twist \bigvee (p_i, q_i)\\
    &= a \twist (\bigvee p_i, \bigwedge q_i)\\
    &= (a \vee \bigvee p_i) \wedge (\neg \bigwedge q_i)\\
    &= (a \vee \bigvee p_i) \wedge (\bigvee \neg q_i) \tag{applying de Morgan}\\
    &= \bigvee (a \vee p_i) \wedge \neg q_i \tag{$\vee$ and $\wedge$ are continuous}\\
    &= \bigvee a \twist (p_i, q_i)
  \end{align*}

  Continuity $\twist$ in its first argument and continuity of $\splus$ follow easily from their definitions and the continuity
  of $\vee$ and $\wedge$.
\end{proof}

\subsection{Fixpoints}

\iterDerivativesN*
\begin{proof}
  \label{prf:iterDerivativesN}
  By induction on $n$. We show the inductive step.
  \begin{align*}
    &\iter_f((n+1) + m)\\
    &=f(\iter_f(n + m)) \tag{definition of $\iter_f$}\\
    &=f(\iter_f(n) \cplus \derive{\iter_f}(n, m)) \tag{by induction}\\
    &=\iter_f(n+1) \cplus \derive{f}(\iter_f(n), \derive{\iter_f}(n, m)) \tag{$f$ is differentiable, definition of $\iter_f$}
  \end{align*}
\end{proof}

\fixpointIter*
\begin{proof}
  \label{prf:fixpointIter}
  \begin{align*}
    &\lfp(\iter_f)\\
    &=\sqcup_{n \in \NN} \iter_f(n)\\
    &=\sqcup_{n \in \NN} \pi_1 (\nextiter_f^n(\bot))
  \end{align*}
\end{proof}

\fixpointPseudoDerivatives*
\begin{proof}
  \label{prf:fixpointPseudoDerivatives}
  Let $\change{w} \in \Delta A$ satisfy \cref{eqn:fixcondition}. Then
  \begin{align*}
    &(f \cplus \change{f})(\fixpoint_A(f) \cplus \change{w})\\
    &= f(\fixpoint(f))
    \cplus
    \adjust(f, \change{f})(\change{w})
    \tag{by \cref{prop:incrementalization}}\\
    &= \fixpoint(f)
    \cplus
    \change{w}
    \tag{rolling the fixpoint and \cref{eqn:fixcondition}}
  \end{align*}

  Hence $\fixpoint(f) \cplus \change{w}$ is a fixpoint of $f \cplus \change{f}$. The converse
  follows from reversing the direction of the proof.
\end{proof}

\iterDerivativesF*
\begin{proof}
  \label{prf:iterDerivativesF}
  The base case is easy to prove.

  For the inductive step:
  \begin{align*}
    &\iter_{n+1}(f \cplus \change{f})\\
    &=(f \cplus \change{f})(\iter_{n}(f \cplus \change{f}))\\
    &= (f \cplus \change{f})(
        \iter_{n}(f)
        \cplus \derive{\iter_{n}}(f, \change{f})
      )
    \tag{ by induction}\\
    &= f(\iter_n(f)) \cplus \derive{\ev}((f, \iter_{n}(f)), (\change{f},
      \derive{\iter_{n}}(f, \change{f})))
    \tag{by \cref{prop:incrementalization}}\\
    & =\iter_{n+1}(f) \cplus \derive{\iter_{n+1}}(f, \change{f})
  \end{align*}
\end{proof}

\leastFixpointDerivatives*
\begin{proof}
  \label{prf:leastFixpointDerivatives}
  $\derive{\iter_n}$ and $\nextiter_{f, \change{f}}$ are continuous since
  $\derive{\ev}$ and $f$ are.

  Hence the set $\{ \derive{\iter}_n \}$ is directed, and so $\sqcup_{i \in \NN}
  \derive{\iter_i}$ is indeed a derivative for $\lfp$.

  We now show that it is equivalent to $\derive{\lfp}$:
  \begin{align*}
    &\sqcup_{n \in \NN} \derive{\iter_n}(f, \change{f})\\
    &=\sqcup_{n \in \NN} \pi_2(\nextiter_{f, \change{f}}^n(\bot))\\
    &=\pi_2(\sqcup_{n \in \NN} \nextiter_{f, \change{f}}^n(\bot)) \tag{$\pi_2$ is continuous}\\
    &= \pi_2 (\lfp(\nextiter_{f, \change{f}})) \tag{$\nextiter_{f, \change{f}}$ is continuous, Kleene's Theorem}\\
    &= \pi_2 ((\lfp(f), \lfp (\lambda\ \change{a}. \derive{\ev}((f, \lfp f), (\change{f}, \change{a})))))
    \tag{by \cref{prop:factoringFixpoints}, and the definition of $\nextiter$}\\
    &= \pi_2 (\lfp(f), \lfp(\adjust(f, \change{f})))\\
    &= \lfp(\adjust(f, \change{f}))\\
    &= \derive{\lfp}(f, \change{f})
  \end{align*}
\end{proof}

\end{document}
