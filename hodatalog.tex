%!TEX root = change-structures.tex

\newcommand{\abs}[2]{\lambda #1.{#2}}

\newcommand{\boolsort}{o}
\newcommand{\intsort}{\iota}
\newcommand{\mng}[1]{\llbracket #1 \rrbracket}
\newcommand{\mmng}[1]{\mathcal{M}\llbracket #1 \rrbracket}
\newcommand{\cmng}[1]{\mathcal{C}\llbracket #1 \rrbracket}
\newcommand{\smng}[1]{\mathcal{S}\llbracket #1 \rrbracket}
\newcommand\makeset[1]{\{#1\}}
\newcommand\disjointTimesS{\disjointTimes_S}
%\newcommand\twistS{\twist_S}
\newcommand\superposeS{{\superpose_S}}
\newcommand{\twistS}{\cplus_{\superpose_S}}
\newcommand\lub{\bigsqcup}

\subsection{Extension to higher-order datalog}

For the syntax of higher-order logic, we use a standard presentation as a simply-typed $\lambda$-calculus with \emph{types}
\[
\sigma \; ::= \; \boolsort \mid \intsort \mid \sigma_1 \to \sigma_2
\]
and \emph{logical constants}, $\neg, \wedge, \vee, \forall_\sigma, \exists_\sigma$, etc., typed as follows:
\[ \neg : \boolsort \to \boolsort \qquad \vee, \wedge : \boolsort \to \boolsort \to \boolsort \qquad \forall_\sigma, \exists_\sigma : (\sigma \to \boolsort) \to \boolsort\]
Standardly we write $\exists_\sigma(\abs{x\!\!:\!\!\sigma}{M})$ as $\exists x\!\!:\!\!\sigma .\, M$; similarly for $\forall_\sigma(\abs{x\!\!:\!\!\sigma}{M})$.
Types of the following shape are called \emph{relational}:
\[
\rho \; ::= \; \intsort \to \boolsort \mid \intsort \to \rho \mid \rho \to \rho'
\]
Note that a relational type has $\boolsort$ as the result type.

\subsubsection{Semantics of types, terms and formulas: standard, monotone and continuous}
The \emph{standard semantics} of types $\smng{\sigma}$ is defined as follows.
\[
\begin{array}{rll}
%  \smng{\mathsf{one}} & \coloneqq  & \makeset{\star} \\  
  \smng{\boolsort} & \coloneqq  & \makeset{0, 1} \\ 
  \smng{\intsort} & \coloneqq  & \mathbb{Z} \\
  \smng{\sigma_1 \to \sigma_2} & \coloneqq  & \smng{\sigma_1} \to \smng{\sigma_2} \quad \hbox{(set-theoretic function space)}
\end{array}
\]
We define the \emph{monotone semantics} $\mmng{\sigma}$ as follows: 
$\mmng{\boolsort} := (\makeset{\bot, \top}, \leq)$; $\mmng{\intsort} := (\mathbb{Z}, =)$; and $\mmng{\sigma_1 \to \sigma_2} := (\mmng{\sigma_1} \to_{\rm m} \mmng{\sigma_2}, \leq)$, the monotone function space, with the extensional ordering.

The \emph{continuous semantics} $\cmng{\sigma}$ is defined similarly: $\cmng{\boolsort} := (\makeset{\bot, \top}, \leq)$; $\cmng{\intsort} := (\mathbb{Z}, =)$; and $\cmng{\sigma_1 \to \sigma_2} := (\cmng{\sigma_1} \to_{\rm c} \cmng{\sigma_2}, \leq)$, the continuous function space.

Let $L$ be a poset. 
Recall that a subset $X \subset L$ is \emph{open} (written $X \in \tau_S$, the Scott topology) if (i) $X$ is up-closed, and (ii) for every $\omega$-increasing chain $(x_i)_{i \in \omega}$, if $\lub_{i \in \omega} x_i$ exists and is in $X$ then $x_i \in X$ for some $i$. 


Let $D$ and $E$ be cpos. 
A function $f:D \to E$ is \emph{continuous} just if it is monotone, and preserves lubs of $\omega$-increasing chains.
Recall that a function $f : D \to E$ is continuous iff it is continuous w.r.t.~the Scott topology.

The semantics of terms and formula-terms are defined standardly. 
\lo{TODO}

\begin{prop}[name=Scott change actions, restate=hodatalog]
 Let $(L, \leq)$ be a poset. Define
  \begin{displaymath}
    \cstr{L}_{\superposeS} 
    \defeq \cstruct{\tau_S
    }{L \disjointTimesS L}{\twistS}
  \end{displaymath}
  where
  \begin{align*}
    L \disjointTimesS L &\defeq \{ (a, b) \in A \times B \mid a \wedge b = \bot \}\\
    a \twist (p, q) &\defeq (a \vee p) \wedge \neg q
  \end{align*}
  and the monoid operator is
  \begin{displaymath}
    (p, q) \splus (r, s) \defeq ((p \wedge \neg s) \vee r, (q \wedge \neg r) \vee s)
  \end{displaymath}
  with identity element $(\top, \bot)$.

  Then $\cstr{L}_\superpose$ is a complete change action on $L$.
\end{prop}

