%!TEX root = change-structures.tex

\newcommand{\abs}[2]{\lambda #1.{#2}}

\newcommand{\boolsort}{o}
\newcommand{\intsort}{\iota}
\newcommand{\mng}[1]{\llbracket #1 \rrbracket}
\newcommand{\mmng}[1]{\mathcal{M}\llbracket #1 \rrbracket}
\newcommand{\cmng}[1]{\mathcal{C}\llbracket #1 \rrbracket}
\newcommand{\smng}[1]{\mathcal{S}\llbracket #1 \rrbracket}
\newcommand\makeset[1]{\{#1\}}
\newcommand\disjointTimesS{\disjointTimes_S}
\newcommand\twistS{\twist_S}
\newcommand\superposeS{{\superpose_S}}

\subsection{Extension to higher-order datalog}

For the syntax of higher-order logic, we use a standard presentation as a simply-typed $\lambda$-calculus with \emph{types}
\[
\sigma \; ::= \; \boolsort \mid \intsort \mid \sigma_1 \to \sigma
\]
and \emph{logical constants}: $\neg, \wedge, \vee, \forall_\sigma, \exists_\sigma$, etc.
\[ \neg : \boolsort \to \boolsort \qquad \forall_\sigma, \exists_\sigma : (\sigma \to \boolsort) \to \boolsort\]
Standardly we write $\exists_\sigma(\abs{x\!\!:\!\!\sigma}{M})$ as $\exists x\!\!:\!\!\sigma .\, M$.
Relational types are defined as:
\[
\rho \; ::= \; \intsort \to \boolsort \mid \intsort \to \rho \mid \rho \to \rho'
\]
Note that relational types are types that have the $\boolsort$ as the result type.

\subsubsection{Semantics of types, terms and formulas: standard, monotone and continuous}
The \emph{standard semantics} of types $\smng{\sigma}$ is defined as follows.
\[
\begin{array}{rll}
%  \smng{\mathsf{one}} & \coloneqq  & \makeset{\star} \\  
  \smng{\boolsort} & \coloneqq  & \makeset{0, 1} \\ 
  \smng{\intsort} & \coloneqq  & \mathbb{Z} \\
  \smng{\sigma_1 \to \sigma_2} & \coloneqq  & \smng{\sigma_1} \to \smng{\sigma_2} \quad \hbox{(set-theoretic function space)}
\end{array}
\]
The \emph{monotone semantics} $\mmng{\sigma}$ is defined similarly: $\mmng{\boolsort} := (\makeset{\bot, \top}, \leq)$; $\mmng{\intsort} := (\mathbb{Z}, =)$; and $\mmng{\sigma_1 \to \sigma_2} := (\mmng{\sigma_1} \to_{\rm m} \mmng{\sigma_2}, \leq)$, the monotone function space.

The \emph{continuous semantics} $\cmng{\sigma}$ is also defined similarly: $\cmng{\boolsort} := (\makeset{\bot, \top}, \leq)$; $\cmng{\intsort} := (\mathbb{Z}, =)$; and $\cmng{\sigma_1 \to \sigma_2} := (\cmng{\sigma_1} \to_{\rm c} \cmng{\sigma_2}, \leq)$, the Scott-continuous function space.

The semantics of terms and formula-terms are defined standardly. 
\lo{TODO}

\begin{prop}[name=Scott change actions, restate=hodatalog]
 Let $L$ be a poset. Define
  \begin{displaymath}
    \cstr{L}_\superposeS \defeq \cstruct{L}{L \disjointTimesS L}{\twistS}
  \end{displaymath}
  where
  \begin{align*}
    L \disjointTimesS L &\defeq \{ (a, b) \in A \times B \mid a \wedge b = \bot \}\\
    a \twist (p, q) &\defeq (a \vee p) \wedge \neg q
  \end{align*}
  and the monoid operator is
  \begin{displaymath}
    (p, q) \splus (r, s) \defeq ((p \wedge \neg s) \vee r, (q \wedge \neg r) \vee s)
  \end{displaymath}
  with identity element $(\top, \bot)$.

  Then $\cstr{L}_\superpose$ is a complete change action on $L$.
\end{prop}

