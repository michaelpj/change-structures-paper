%!TEX root = change-structures.tex

\newcommand{\abs}[2]{\lambda #1.\,{#2}}

\newcommand{\boolsort}{o}
\newcommand{\intsort}{\iota}
\newcommand{\mng}[1]{\llbracket #1 \rrbracket}
\newcommand{\mmng}[1]{\mathcal{M}\llbracket #1 \rrbracket}
\newcommand{\cmng}[1]{\mathcal{C}\llbracket #1 \rrbracket}
\newcommand{\smng}[1]{\mathcal{S}\llbracket #1 \rrbracket}
\newcommand{\makeset}[1]{\{#1\}}
\newcommand{\disjointTimesS}{\disjointTimes_S}
\newcommand{\superposeS}{{\superpose_S}}
\newcommand{\twistS}{\cplus_{\superpose_S}}
\newcommand{\lub}{\bigsqcup}
\newcommand{\truetm}{\mathsf{true}}
\newcommand{\falsetm}{\mathsf{false}}
\newcommand{\cTo}{\Rightarrow_{\rm c}}
\newcommand{\mTo}{\Rightarrow_{\rm m}}
\newcommand{\mexistsfn}{\mathsf{mexists}}
\newcommand{\mfunc}{T^\mathcal{M}}
\newcommand{\impliesfn}{\mathsf{implies}}

\newcommand{\andfn}{\mathsf{and}}
\newcommand{\mandfn}{\mathsf{mand}}
\newcommand{\andterm}{\mathsf{and}}
\newcommand{\orfn}{\mathsf{or}}
\newcommand{\morfn}{\mathsf{mor}}
\newcommand{\notfn}{\mathsf{not}}
\newcommand{\existsfn}{\mathsf{exists}}
\newcommand{\forallfn}{\mathsf{forall}}
\newcommand{\dom}{\mathrm{dom}}

\subsection{Extension to higher-order Datalog}

\lo{TODO: Check that the following definitions are consistent extensions of Datalog in the preceding.}

\subsubsection{Syntax} 
We use a standard presentation of higher-order logic as a simply-typed $\lambda$-calculus with \emph{types}
\[
\sigma \; ::= \; \boolsort \mid \intsort \mid \sigma_1 \to \sigma_2
\]
generated from the base types $\boolsort$ (booleans) and $\intsort$ (individuals).
Types of the following shape are called \emph{relational}:
\[
\rho \; ::= \; \intsort \to \boolsort \mid \intsort \to \rho \mid \rho \to \rho'
\]
Note that a relational type has $\boolsort$ as the result type.
\emph{Terms} are just the standard simply-typed $\lambda$-terms generated from a given first-order signature $\Sigma$, and the following set of \emph{logical constants}, typed as follows:
\[ \neg : \boolsort \to \boolsort \qquad \vee, \wedge, \Rightarrow : \boolsort \to \boolsort \to \boolsort \qquad \forall_\sigma, \exists_\sigma : (\sigma \to \boolsort) \to \boolsort\]
Standardly we write $\exists_\sigma(\abs{x\!\!:\!\!\sigma}{M})$ as $\exists x\!\!:\!\!\sigma .\, M$; similarly for $\forall_\sigma(\abs{x\!\!:\!\!\sigma}{M})$.
\emph{Formulas} are just terms of type $\boolsort$.
We will write variables generally using $x,y,z$, or $X,Y,Z$ when we want to emphasise that they are of higher-order types.
Given a typing environment $\Delta$ (i.e.~a function from $\dom(\Delta)$ to types), \emph{typing judgements} have the form $\Delta \vdash s : \sigma$, and validity is defined by induction over the standard typing rules.

\paragraph{Horn clauses}
Fix a typing environment $\Delta$ of relational variables (i.e.~variables of relational types). 
The \emph{goal formulas} over $\Delta$, typically $G$, and the \emph{definite formulas} over $\Delta$, typically $D$, are the subset of all formulas defined by induction:
\[
  \begin{array}{rcl}
    G & \Coloneqq & M \mid G \wedge G \mid G \vee G \mid \exists x\!\!:\!\!\sigma.\: G  \\
    D & \Coloneqq & \truetm \mid \forall x\!\!:\!\!\sigma.\: D \mid D \wedge D \mid G \Rightarrow X \, \overline x \\
  \end{array}
\]
in which $\sigma$ is either the type of individuals $\intsort$ or a relational type $\rho$, $M$---called \emph{atom}---is an applicative formula of shape ${X \, M_1 \cdots M_k}$
in which $X$ is a relational variable and each $M_i$ is a term.
%\footnote{We do not require the head variable of $M$ to be in $\dom(\Delta)$.},
In the last alternative, $X \in \dom(\Delta)$ and $\overline{x} = x_1 \cdots x_n$ a sequence of pairwise distinct variables.

\subsection{Continuous semantics}

\begin{figure*}
  \fbox{
    \begin{minipage}[t]{0.9\textwidth}
\[
  \begin{array}{rcl}
    \cmng{\Delta \vdash x : \rho}(\alpha) &=& \alpha(x) \\
    \cmng{\Delta \vdash c : \rho}(\alpha) &=& c^\rho \\
    %\cmng{\Delta \vdash \phi : \boolsort}(\alpha) &=& \smng{\Delta \vdash \phi : \boolsort}(\alpha) \\
    \cmng{\Delta \vdash {G \, H} : \rho_2}(\alpha) &=& \cmng{\Delta \vdash G : \rho_1 \to \rho_2}(\alpha)(\cmng{\Delta \vdash H : \sigma_1}(\alpha)) \\
    \cmng{\Delta \vdash {G \, N} : \rho}(\alpha) &=& \cmng{\Delta \vdash G : \iota \to \rho}(\alpha)(\cmng{\Delta \vdash N : \iota}(\alpha)) \\
    \cmng{\Delta \vdash \abs{x:\sigma}{G} : \sigma}(\alpha) &=& %\abs{x \in \cmng{\sigma}}{\cmng{\Delta, x:\sigma \vdash G : \sigma}(\alpha)} \\
    \abs{x' \in \cmng{\sigma}}{\cmng{\Delta, x:\sigma \vdash G : \sigma}(\alpha[x \mapsto x'])}
  \end{array}
\]
\end{minipage}
}
\caption{Continuous semantics of terms.}\label{fig:cts-term-semantics}
\vspace{-12pt}
\end{figure*}

Let $L$ be a poset. 
Recall that a subset $X \subseteq L$ is \emph{open} (written $X \in \tau_S$, the Scott topology) if (i) $X$ is up-closed, and (ii) for every $\omega$-increasing chain $(x_i)_{i \in \omega}$, if $\lub_{i \in \omega} x_i$ exists and is in $X$ then $x_i \in X$ for some $i$. 
We write $\overline{\tau_S} := \makeset{\overline X \mid X \in \tau_S}$ for the collection of closed sets.


Let $D$ and $E$ be cpos. 
A function $f:D \to E$ is \emph{continuous} just if it is monotone, and preserves lubs of $\omega$-increasing chains.
Recall that a function $f : D \to E$ is continuous iff it is continuous w.r.t.~the Scott topology.

\paragraph{Continuous type frame.}
We define the \emph{continuous type frame} by induction:
\[
   \cmng{\iota} \coloneqq (A_\iota, =)
   \qquad\quad \cmng{\boolsort} \coloneqq (\mathbbm{2}, \leq)
   \qquad\quad \cmng{\sigma_1 \to \sigma_2} \coloneqq (\cmng{\sigma_1} \cTo \cmng{\sigma_2}, \leq)
\]
where $X \cTo Y$ is the continuous function space between cpos $X$ and $Y$, which is itself a CPO with respect to the pointwise ordering $\leq$.
Of course, in case $X$ is the discrete poset $A_\iota$, this coincides with the full function space.
%We extend the lattice structure of $\mathbbm{2}$ to all relations $\cmng{\rho}$, analogously to the case of the full function space (and we reuse the same notation since there will be no confusion);
Define $\cmng{\Delta} \coloneqq \Pi x \in \dom(\Delta).\,\cmng{\Delta(x)}$.

Note that an element $f$ of $X_1 \cTo \cdots{} \cTo X_n \cTo \mathbbm{2} \cong X_1 \times \cdots \times X_n \cTo \mathbbm{2}$ is the \emph{characteristic function} of the (higher-order) relation $f^{-1}(1) \subseteq X_1 \times \cdots \times X_n$, which is always Scott open.


\paragraph{Continuous interpretation.}
We interpret a typing environment $\Delta$ by the indexed product:
$
  \cmng{\Delta} \coloneqq \Pi x \in \dom(\Delta).\cmng{\Delta(x)}
$,
that is, the set of all functions on $\dom(\Delta)$ that map $x$ to an element of $\cmng{\Delta(x)}$;
these functions, typically $\alpha$, are called \emph{valuations}.
The interpretation of a term $\Delta \vdash M : \sigma$ is a function $\cmng{\Delta \vdash M : \sigma}$ that belongs to the set $\cmng{\Delta} \cTo \cmng{\sigma}$, and which is defined in Figure~\ref{fig:cts-term-semantics}.
%As for the standard interpretation, we assume a fixed interpretation $A$ of the background theory, which is left implicit in the notation.
The logical constants $c : \rho$ are interpreted by the following functions:
\[
  \begin{array}{cc}
  \begin{array}{rcl}
    \orfn(b_1)(b_2) &=& \max \{b_1,b_2\} \\
    \andfn(b_1)(b_2) &=& \min \{b_1,b_2\} \\
    \existsfn_\sigma(f) &=& \max \{ f(v) \mid v \in \cmng{\sigma} \}
  \end{array}
  &
  \begin{array}{rcl}
    \notfn(b) &=& 1 - b \\
    \impliesfn(b_1)(b_2) &=& \orfn(\notfn(b_1))(b_2) \\
    \forallfn_\sigma(f) &=& \notfn (\existsfn_\sigma(\notfn \circ f))
  \end{array}
  \end{array}
\]
Whilst the standard interpretation of the positive logical constants for conjunction and disjunction will suffice, the interpretation of existential quantification needs to be relativised to the monotone setting:
$
  \mexistsfn_\sigma(r) = \mathit{max} \{r(d) \mid d \in \cmng{\sigma}\}
$.
As for standard semantics, we write $\cmng{G}$ for $\cmng{\Delta \vdash  G : \rho}$, whenever the judgement $\Delta \vdash  G : \rho$ is clear from the context.

%Since the implication function $\impliesfn$ is not monotone (in its first argument), definite formulas are not interpretable in a monotone frame.
%However, it is possible to interpret logic programs.
% To that end, we define the functional $\mfunc_{P:\Delta}$ on semantic environments by: $\mfunc_{P:\Delta}(\alpha)(x) = \cmng{\Delta \vdash  P(x) : \Delta(x)}(\alpha)$.
% In analogy with the Horn clause problem, we call a prefixed point of $\mfunc_{P:\Delta}$ a \emph{model} of the program $P$.
% This construction preserves the logical order.



% \subsubsection{Semantics of types, terms and formulas: standard, monotone and continuous}
% The \emph{standard semantics} of types $\smng{\sigma}$ is defined as follows.
% \[
% \begin{array}{rll}
% %  \smng{\mathsf{one}} & \coloneqq  & \makeset{\star} \\  
%   \smng{\boolsort} & \coloneqq  & \makeset{0, 1} \\ 
%   \smng{\intsort} & \coloneqq  & \mathbbm{Z} \\
%   \smng{\sigma_1 \to \sigma_2} & \coloneqq  & \smng{\sigma_1} \to \smng{\sigma_2} \quad \hbox{(set-theoretic function space)}
% \end{array}
% \]
% We define the \emph{monotone semantics} $\cmng{\sigma}$ as follows: 
% $\cmng{\boolsort} := (\makeset{\bot, \top}, \leq)$; $\cmng{\intsort} := (\mathbbm{Z}, =)$; and $\cmng{\sigma_1 \to \sigma_2} := (\cmng{\sigma_1} \to_{\rm m} \cmng{\sigma_2}, \leq)$, the monotone function space, with the extensional ordering.

% The \emph{continuous semantics} $\cmng{\sigma}$ is defined similarly: $\cmng{\boolsort} := (\makeset{\bot, \top}, \leq)$; $\cmng{\intsort} := (\mathbbm{Z}, =)$; and $\cmng{\sigma_1 \to \sigma_2} := (\cmng{\sigma_1} \to_{\rm c} \cmng{\sigma_2}, \leq)$, the continuous function space.

\begin{prop}[name=Scott change actions, restate=hodatalog]
\lo{TODO: Find better notations.}
 Let $(L, \leq)$ be a poset. Define
  \begin{displaymath}
    \cstr{L}_{\superposeS} 
    \defeq \cstruct{\tau_S
    }{L \disjointTimesS L}{\twistS}
  \end{displaymath}
  where
  \begin{align*}
    L \disjointTimesS L &\defeq \{ (a, b) \in \tau_S \times \overline{\tau_S} \mid a \cap b = \emptyset \}\\
    a \twistS (p, q) &\defeq (a \cup p) \cap \overline q
  \end{align*}
  and the monoid operator is
  \begin{displaymath}
    (p, q) \splus (r, s) \defeq ((p \cap \overline s) \cup r, (q \cap \overline r) \cup s)
  \end{displaymath}
  with identity element $(L, \emptyset)$.

  Then $\cstr{L}_\superpose$ is a complete change action on $L$.
\end{prop}

\lo{Let $a, b \in \tau_S$. Then $a \twistS (b, \overline b) = b$.}

\lo{There is a version of Theorem~\ref{thm:concreteDatalog} for higher-order Datalog.}
